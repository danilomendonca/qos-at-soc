\section{Amea\c cas a validade}

As poss\'{i}veis amea\c{c}as a aos resultados dessa pesquisa, bem como a generaliza\c c\~{a} dos mesmos, s\~{a}o descritas nessa se\c c\~{a}o, conforme a classifica\c c\~{a}o apresentada em~\cite{leedy1980practical}. 

\noindent
\emph{A Validade de constru\c c\~{a}o} est\'{a} relacionada
\`{a}s decis\~{o}es operacionais que foram tomadas, durante o
planejamento do estudo, com o intuito de responder \`{a}s quest\~{o}es de
pesquisa. As principais constru\c{c}\~{o}es do nosso estudo est\~{a}o em definir os conceitos de ``QoS em SOC'', definir o per\'{i}odo de publica\c{c}\~{o}es significativa para o estudo e \textbf{o mapeamento do estudo em si}. Conceitos de SOC:


Como nosso objetivo \'{e} identificar tend\^{e}ncias de pesquisa na \'{a}rea de QoS em SOC, o fato de n\~{a}o termos considerado trabalhos publicados ap\'{o}s Dezembro de 2011 pode representar uma amea\c ca aos nossos resultados. Por outro lado, consideramos que trabalhos publicados entre Janeiro de 2009 e Dezembro de 2011 permitem-nos avaliar um per\'{i}odo bastante significativo do estudo, conforme ilustrado na Figura~\ref{fig:barplotAnoPublicacoes}. Al\'{e}m disso, consideramos que uma avalia\c c\~{a}o parcial do ano corrente (2012) poderia levar a avalia\c{c}\~{o}es inconclusivas quanto \`{a} tend\^{e}ncia do estudo. Por fim, quanto \`{a} terceira amea\c{c}a de constru\c{c}\~{a}o, para realizar nosso estudos seguimos as diretrizes de ref\^{e}ncia de Kitchenham~\cite{kitchenham:techReport2007} para definir as quest\~{o}es de pesquisa, crit\'{e}rio de busca e protocolo do estudo.

\noindent
\emph{A Validade interna}
Estabelece a rela\c{c}\~{a}o causal garantindo que certas condi\c{c}\~{o}es levem de fato a outras condi\c{c}\~{o}es. Em particular, a ameaça nesse sentido consiste na seleção incompleta ou inadequada das contribuições ou a tendenciosidade da visão individual de cada pesquisador envolvido. Um estudo abrangente como este, envolvendo cinco pesquisadores, leva a algum risco nesse sentido, uma vez que a classifica\c c\~{a}o dos artigos pode envolver certo grau subjetividade na interpreta\c{c}\~{a}o de cada faceta. Com o intuito de minimizar tal amea\c{c}a, algumas reuni\~{o}es objetivaram esclarecer as facetas do estudo, enquanto que outras serviram para realizar avalia\c c\~{o}es em pares. Foram feitas diversas discuss\~{o}es entre os autores deste trabalho sempre que a an\'{a}lise de um artigo resultava em d\'{u}vidas quanto sua classifica\c c\~{a}o. Finalmente, a seleção correta pode ser percebida tanto definição adequada do protocolo na Seção~\ref{sec:review_method} quanto por resultados conclusivos como aqueles apresentados por principais grupos na Se\c{c}\~{a}o ~\ref{sec:QP3}.

\noindent
\emph{A Validade externa} est\'{a} relacionada com a possibilidade de
generaliza\c c\~{a}o do nosso estudo. Esse \'{e} um aspecto muito
importante que est\'{a} relacionado ao escopo do nosso estudo, que tem
como objetivo entender a pesquisa realizada em QoS / SOC. Por outro
lado, o termo QoS \'{e} bastante amplo, sendo invi\'{a}vel realizar
um estudo compreensivo sobre todos os atributos de qualidade
relacionados a SOC (seguran\c ca, performance, toler\^{a}ncia a falhas
etc.). Nesse sentido, e conforme discutido na se\c{c}\~{a}o~\ref{sec:review_method}, a nossa \texttt{string} de busca se restringiu aos termos \emph{quality of services}, \emph{quality of service} e \emph{QOS}. Tais termos deveriam aparecer em algum campo de metadados de um artigo (como t\'{i}tulo, resumo ou palavras-chave) para que o mesmo fosse recuperado pelo \emph{crawler} desenvolvido. Ou seja, n\~{a}o generalizamos nossos resultados para pesquisas desenvolvidas em uma \'{a}rea espec\'{i}fica de QOS, mas sim para a \'{a}rea de QOS de forma mais gen\'{e}rica. Como
trabalho futuro, temos o interesse de reproduzir mapeamentos sistem\'{a}ticos de estudo para algumas \'{a}reas espec\'{i}ficas de QOS (tratamento de exce\c
c\~{o}es, seguran\c ca etc). 

\noindent
\emph{A confiabilidade} est\'{a} relacionada ao grau de objetividade de
um estudo, em que uma poss\'{i}vel replica\c c\~{a}o levaria a
resultados parecidos. Com a infraestrutura que disponibilizamos para a realiza\c{c}\~{a}o de mapeamentos sistem\'{a}ticos de estudos, podemos tanto estender essa investiga\c c\~{a}o em um trabalho futuro quanto convidarmos outros pesquisadores para replicar tal avalia\c c\~{a}o. Em particular, a estrat\'{e}gia de s\'{i}ntese e interpreta\c{c}\~{a}o escolhidas podem trazer diferentes \emph{insights}, mas acreditamos que as tend\^{e}ncias observadas devem permanecer as mesmas.