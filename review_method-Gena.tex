%TCIDATA{LaTeXparent=0 0 sbrc_2013.tex}
%Review Method

\section{M\'{e}todo do Estudo}\label{sec:review_method}

\subsection{Protocolo do Estudo}

\subsection{Quest\~{o}es de Pesquisa}

As quest\~{o}es de pesquisa foram organizadas de acordo com a motiva\c{c}\~{a}o desse estudo, que \'{e} investigar e categorizar as contribui\c{c}\~{o}es de pesquisa em computa\c{c}\~{a}o orientada a servi\c{c}o no contexto de qualidade de servi\c{c}o. Esse estudo tem como objetivo responder \`{a}s seguintes perguntas: 

\begin{itemize}
\item {\bf QP1} Qual o interesse de pesquisa da comunidade cient\'{i}fica no tema em quest\~{a}o? 
\item {\bf QP2}Quais s\~{a}o as \'{a}reas de COS mais frequentemente consideradas no contexto de QoS atualmente e quais s\~{a}o os pesquisadores com maior n\'{u}mero de contribui\c{c}\~{o}es?
\item {\bf QP3} Quais os atributos de qualidade frequentemente considerados nos estudos?
\item {\bf QP4} Qual o foco da contribui\c{c}\~{a}o de pesquisa realizada?   
\end{itemize}

%Explicar o sentido de cada QP individualmente...
A QP1 almeja identificar o estado da pesquisa relacionada a COS no contexto de qualidade de servi\c{c}os em termos quantitativos, ou seja, apontar o n\'{u}mero de pubca\c{c}\~{o}es na \'{a}area e os principais autores envolvidos.

A QP2 tem como objetivo trazer uma perspectiva do cen\'{a}rio das pesquisas em Computa\c{c}\~{a}o Orientada a Servi\c{c}o com foco em QoS atualmente. Para responder a essa pergunta, precisaremos delimitar primeiramente qual o per\'{i}odo que tenha uma representa\c{c}\~{a}o significativa a ser analisada, com base na frequ\^{e}ncia do n\'{u}mero de publica\c{c}\~{o}es principalmente nos \'{u}ltimos anos. Em segundo lugar, ainda com rela\c{c}\~{a}o \`{a} primeira quest\~{a}o de pesquisa, precisamos definir quais s\~{a}o as \'{a}reas que melhor caracterizam as diversas contribui\c{c}\~{o}es de pesquisa em COS. 

Com rela\c{c}\~{a}o \`{a} QP3, pretendemos obter com esse estudo quais s\~{a}o os atributos de QoS mais frequentemente explorados em COS. Em outras palavras, considerando que QoS, nesse contexto, envolve atributos como disponbilidade, confiabilidade, desempenho, seguran\c{c}a, escalabilidade, custo, SLA, quais desses atributos de QoS est\~{a}o de fato em foco.

A QP4 tem como objetivo identificar o tipo de pesquisa que est\'{a} sendo conduzido pelas contribui\c{c}\~{o}es nesse contexto. Vale ressaltar aqui, que n\~{a}o pretendemos avaliar o m\'{e}rito da pesquisa em si. Est\'{a} fora do escopo desse estudo avaliar tal quest\~{a}o.

\subsection{Estrat\'{e}gia de Busca}
%Our search strategy consisted of both manual and electronic search. Electronic search was performed in the following digital databases: ACM Digital Library, CiteSeerX, Compendex, Google Scholar, IEEE Xplore and SpringerLink. These are relevant electronic databases to computer science and software engineering, also used in a number of systematic studies in the area [2] .To formulate the search string for electronic database search, we used an approach suggested by Kitchenham [1]. The strategy derives the search string from the research questions using a composition with Boolean operators OR and AND. Table 1 presents the search string. The justification for using manual search was that creativity in RE is a relatively novel area, therefore manual search in conferences and journals provided extra confidence that relevant papers would be found.  To have a more representative set of studies, the “snow-balling" technique was adopted [3], in which the references of the identified papers were analyzed.
Nossa estrat\'{e}gia de busca consistiu essencialmente na busca eletr\^{o}nica nas seguintes bibliotecas digitais: ACM Digital Library, ScienceDirect, IEEE Xplore e SpringerLink. Essas est\~ao entre as bibliotecas mais relevantes para a Ci\^{e}ncia da Computa\c{c}\~ao. Para formular os termos de busca para a base de dados eletr\^{o}nica, usamos a abordagem sugerida por Kitchenham\cite{Barbara Kitchenham}. A estrat\'{e}gia deriva os termos de busca a partir das quest\~{o}es de pesquisa  usando uma composi\c{c}\~{a}o com os operadores OR e AND. Para evitar a tendenciosidade quanto a quais comunidades de pesquisa mais atuantes no contexto desse estudo, assim como para obter um tamanho real das contribui\c{c}\~{o}es e seus grupos de pesquisa, resolvemos n\~ao adotar a t\'{e}cncia de \emph{snow-balling}, onde outros trabalhos relacionados podem ser encontrados a partir das refer\^{e}ncias dos trabalhos extraídos automaticamente \cite{VIDE Using Mapping Studies in Software Engineering (Proc. of PPIG 2008}

\subsection{Crit\'{e}rio de Inclus\~{a}o e Exclus\~{a}o}

\subsection{Sele\c{c}\~{a}o do Estudo}
Falar sobre o \emph{crawler} (automatiza\c{c}\~ao da busca) em cada biblioteca digital e como foram armazenados em um site e configurados para a revis\~ao colaborativa de todos os autores desse trabalho. 

Falar do n\'{u}mero total de artigos inicialmente e quantos foram filtrados ao final, em virtude do criterio de exclus\~ao. Mencionar onde iremos disponbilizar a base dos artigos incluídos ao final.

\subsection{Extra\c{c}\~{a}o de Dados e An\'{a}lise}

Descrever como o ambiente no Heroku est\'{a} organizado (facets), explicar os termos principalmente os de Computa\c{c}\~ao Orientada a Servi\c{c}os qual a refer\^{e}ncia de significado que usamos. ``Each author individually extracted data from a subset of papers. We jointly discussed unclear issues and solved discrepancies in the analysis.'' Os resultados tambem foram gerados automaticamente por meio da propria ferramenta....

