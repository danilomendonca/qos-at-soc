\documentclass[12pt]{article}

\usepackage{sbc-template}

\usepackage{graphicx,url}

\usepackage[T1]{fontenc}
\usepackage[brazilian,portuges]{babel}
\usepackage[utf8]{inputenc}

\sloppy

\title{Estudo Sistem\'{a}tico de Computa\c{c}\~{a}o Orientada a Servi\c{c}os no Contexto de QoS}
\author{Danilo Filgueira Mendon\c{c}a\inst{1}, Gena\'{i}na Nunes Rodrigues\inst{1}, \\ Rodrigo Bonif\'{a}cio\inst{1}, Alet\'{e}ia Favacho\inst{1}, Maristela Holanda\inst{1} }

\address{Department of Computer Science -- University of Brasilia (UnB) -- Campus Darcy Ribeiro\\
70910-900, Brasilia, DF, Brazil
\email{dfmendonca@gmail.com, \{genaina, rbonifacio, aleteia, mholanda\}@cic.unb.br}
}

\begin{document} 

\maketitle

\begin{abstract}
  In the last years, the field of service oriented computing (SOC) has received a growing interest from researchers and practitioners, particularly with respect to quality of service (QoS). This paper presents a mapping study to aggregate literature in this field in order to find trends and research opportunities regarding QoS in SOC. We analysed 250 papers and we were able to identify major groups that have contributed to different aspects in the context of our study. We also show that, with respect to SOC contributions dealing with QoS properties, most of them concentrate on runtime issues, such as monitoring and adaptation. Regarding quality attributes, a vast majority of the papers use generic models, so that the proposed solutions are independent of the particularities of a quality attribute. In spite of that, we find that availability, performance and reliability were the major highlights. With respect to research type, most of the reviewed studies propose new solutions, instead of evaluating and validating existing proposals--- a symptom of a field that does not follow established research paradigms.
\end{abstract}

\begin{resumo}
  Nos últimos anos, o campo da computação orientada a serviços (SOC) recebeu um crescente interesse de pesquisadores e praticantes, particularmente no que diz respeito a qualidade de serviços (QoS). Esse artigo apresenta um estudo de mapeamento para agrupar a literatura neste campo. Nossas maiores descobertas mostram que, em relação às contribuições para SOC envolvendo propriedades de QoS, a maioria se concentra em monitoramento e adaptação, enquanto poucas focam em coordenação e comunicação. A respeito dos atributos de qualidade, uma vasta maioria dos artigos utilizam modelos genéricos, de modo que suas contribuições são independentes das particularidades de um dado atributo. Não obstante, descobrimos que disponibilidade e desempenho foram os principais destaques, ao contrário de custo, escalabilidade e segurança. A respeito do tipo de pesquisa, a maioria dos estudos avalizados propuseram novas soluções em vez de avaliarem ou validarem propostas existentes --- um sintoma de um campo que não segue paradigmas de pesquisa estabelecidos.
\end{resumo}

\newcommand{\AllPubs}{1239}
\newcommand{\AcceptedPubs}{364}
\newcommand{\Ciclodevida}{13.46 \%}
\newcommand{\Composicao}{28.85 \%}
\newcommand{\MonitoramentoAdaptacao}{30.49 \%}
\newcommand{\DescobrimentoeSelecao}{30.77 \%}
\newcommand{\ModelosdeQoSeLinguagens}{31.32 \%}
\newcommand{\CoordenacaoeComunicacao}{8.24 \%}
\newcommand{\Outros}{10.44 \%}
\newcommand{\Custo}{14.01 \%}
\newcommand{\Confiabilidade}{27.20 \%}
\newcommand{\Disponibilidade}{28.57 \%}
\newcommand{\Desempenho}{37.36 \%}
\newcommand{\SLA}{56.59 \%}
\newcommand{\Escalabilidade}{6.04 \%}
\newcommand{\Seguranca}{9.34 \%}
\newcommand{\Pessoal}{0.27 \%}
\newcommand{\Experiencia}{1.37 \%}
\newcommand{\Validacao}{45.05 \%}
\newcommand{\Avaliacao}{9.07 \%}
\newcommand{\Solucao}{90.66 \%}


%TCIDATA{LaTeXparent=0 0 sbrc_2013.tex}
%Introdução

\section{Introdu\c{c}\~{a}o}\label{sec:introduction}

A \emph{Computa\c c\~{a}o Orientada a Servi\c cos} (ou \emph{Service Oriented Computing} -- SOC) emergiu como 
um novo paradigma de computa\c c\~{a}o que utiliza servi\c cos como componentes b\'{a}sicos para o desenvolvimento 
de aplica\c c\~{o}es~\cite{papazoglou:cacm2003}, e tendo como blocos b\'{a}sicos as opera\c c\~{o}es de infraestrutura providas 
pela arquiteturas orientadas a servi\c co (como publica\c c\~{a}o, sele\c c\~{a}o, descoberta e composi\c c\~{a}o).  
Um dos objetivos em SOC \'{e} a automa\c c\~{a}o e integra\c c\~{a}o de processos de neg\'{o}cio e/ou cient\'{i}ficos que podem envolver 
diferentes organiza\c c\~{o}es. Dada as necessidades de orquestra\c c\~{a}o e coreografia e \`{a}s  caracter\'{i}ticas tipicamente distribu\'{i}das e heterog\^{e}neas encontradas 
nos cen\'{a}rios aos quais SOC foi proposta, novas preocupa\c c\~{o}es relacionadas \`{a} garantia de qualidade de 
servi\c co (\emph{Quality of Service} --- QoS) motivaram um interesse crescente dos 
pesquisadores na \'{a}rea. 

Quando novas \'{a}reas de pesquisa emergem, particularmente quando amparadas por 
certo apelo da ind\'{u}stria, \'{e} natural que uma quantidade significativa de 
contribui\c c\~{o}es cient\'{i}ficas foquem primariamente em novas propostas de solu\c c\~{a}o 
(fases que compreendem o momento da formula\c c\~{a}o de um paradigma por uma comunidade ainda restrita 
ao momento do seu refinamento e explora\c c\~{a}o por uma audi\^{e}ncia mais ampla~\cite{redwine:icse1985}).   
Por outro lado, com o amadurecimento da pesquisa realizada em uma \'{a}rea, espera-se o aprofundamento 
em termos de evid\^{e}ncias relacionadas \`{a} aplicabilidade das t\'{e}cnicas propostas, para que em seguida as mesmas 
sejam adotadas pela comunidade--- Redwine et al. argumenta que 
esse ciclo de \emph{matura\c c\~{a}o} dura aproximadamente 20 anos para a \'{a}rea de tecnologia, em 
particular para a \'{a}rea de software~\cite{redwine:icse1985}.

Apesar da pesquisa em SOC ter emergido h\'{a} pouco mais de 10 anos 
(uma edi\c c\~{a}o especial da \emph{Communications of ACM}~\cite{papazoglou:cacm2003} sobre SOC foi publicada em 2003), 
n\~{a}o existe um mapeamento que possibilite identificar o est\'{a}gio da pesquisa realizada na \'{a}rea de qualidade de 
servi\c cos em SOC, dificultando a identifica\c c\~{a}o de tend\^{e}ncias e oportundiades de pesquisa e tornando mais 
lenta a ado\c c\~{a}o das t\'{e}cnicas propostas. Nossa hip\'{o}tese inicial \'{e} que o foco de pesquisa em QoS / SOC deve 
estar alcan\c cando o patamar de amadurecimento em que uma quantidade significativa de trabalhos objetiva evidenciar os 
benef\'{i}cios das solu\c c\~{o}es propostas por meio de estudos experimentais e valida\c c\~{o}es. Com o intuito de investigar tal hip\'{o}tese, 
este artigo apresenta os resultados de um mapeamento sistem\'{a}tica de estudos (cujo protocolo \'{e} descrito na Se\c c\~{a}o~\ref{sec:review_method}) 
sobre a \'{a}rea, revelando, al\'{e}m de da maturidade da pesquisa, quais s\~{a}os os principais atributos de qualidade 
investigados e grupos de pesquisa que atuam na \'{a}rea. 

Organizamos nossa investiga\c{c}\~{a}o em termos de quatro quest\~{o}es de pesquisa (apresentadas inicialmente na 
Se\c c\~{a}o~\ref{sec:questoesPesquisa}). Tais quest\~{o}es s\~{a}o respondidas ao longo de toda a Se\c c\~{a}o~\ref{sec:resultados}. 
Finalmente, consolidamos nossas observa\c c\~{o}es e apresentamos as amea\c cas para a validade do estudo na 
Se\c c\~{a}o~\ref{sec:conclusao}. 
 
 


% SOC, SOA, \textit{Web Services}, SOAP, REST, Orientação a Serviços, Computação em Nuvem. Nos últimos anos esses e outros acrônimos tornaram-se frequentes na tecnologia da informação. O surgimento de um novo paradigma, impulsionado pelo amadurecimento da internet e pela proximidade crescente entre negócios e TI, criou novos caminhos e oportunidades para trabalhos de desenvolvimento e pesquisa. Nesse sentido, um grande número de estudos foram e vem sendo conduzidos com foco nos diversos aspectos da computação orientada a serviços, tais quais arquitetura, modelos, métodos, processos, ferramentas diversas, frameworks, métricas, problemas solucionados e ainda vigentes. Desta forma, a intenção daqueles interessados em iniciar suas atividades na área fica comprometida pela dificuldade em se obter informações claras sobre o atual estado da arte, os desafios e os temas mais abordados e aqueles com deficit de pesquisas. Esses dados são cruciais para que esforços sejam bem direcionados e para que a ciência caminhe em cooperação e com eficiência.

% Um mapeamento sistemático de estudos visa classificar de forma sistemática e ampla um conjunto de estudos. Dada a grande quantidade de publicações no escopo da orientação a serviços, sua metodologia ágil e que permite a análise de um maior número de estudos justifica sua escolha em detrimento de outras metodologias, como o \textit{Systematic Literature Review} \cite{Petersen_Feldt_Mujtaba_Mattsson_2007}. Essa última exige uma análise minuciosa e detalhada de cada publicação, o que requer um esforço considerável e inviabiliza a inclusão de um grande número de publicações num quadro de poucos pesquisadores. Assim, dados os fatos citados e o interesse em se obter uma classificação ampla e significativa da ciência relacionada à orientação a serviços, de caráter inicial e que irá servir de subsídio a outros estudos, este trabalho de conclusão de curso em Engenharia de Redes de Comunicação realiza um mapeamento sistemático de estudos abrangendo a orientação a serviços. 

% Segundo \cite{Papazoglou:2007:SOA:1265289.1265298}, devido ao crescente acordo na implementação e gerência de aspectos funcionais de serviços, tal qual a adoção de WSDL para a descrição, SOAP para troca de mensagens, ou WS-BPEL para a composição, os interesses de pesquisadores estão se voltando aos aspectos não funcionais de aplicações orientadas a serviços. Visando essa constatação, nosso mapeamento irá concentrar-se na questão de qualidade, ou aspectos não funcionais, sobretudo a qualidade de serviços, termo aqui empregado de forma literal e posterior ao termo QoS, uma vez que os principais agentes do paradigma em questão são, coincidentemente, denominados serviços. Ademais, o ambiente proposto pelo SOC está sujeito a condições particulares diferentes daquelas já estudadas e conhecidas em outros paradigmas, havendo variáveis que elevam a complexidade da análise de parâmetros de qualidade, tanto na fase de planejamento quanto em fase de execução por meio do monitoramento e da gerência dos serviços, sendo esse um obstáculo sólido à adoção de arquiteturas como o SOA. Nesse sentido, o presente estudo visa mapear as publicações relacionadas a essas questões, contemplando cenários com ou sem o uso de SOA, proporcionando uma redução da incerteza quanto ao atual estado de desenvolvimento da ciência contribuinte ao tema abordado e quanto aos desafios e avanços já conquistados.

% No que tange a trabalhos relacionados, este trabalho possui características inéditas dentro do campo de QoS em SOC. Dentre as referências atuais e mais relevantes no que concerne modelos de QoS em SOC pode ser encontrado em \cite{pernici}, produzido pelo projeto europeu S-CUBE~\cite{scube}. Esta, além desse citado, produziu uma coletânea de outros relatórios e trabalhos que analisam publicações em praticamente todas as eferas do SOC. Entretanto, trata-se de trabalhos de Systematic Literature Review, visto que analisam profundamente as publicações envolvidas na área e as restringe àquelas com maior qualidade e aceitação, indicando as vantagens e limitações das propostas analisadas. Em contraste, o MS proposto abrange um número maior de estudos, trazendo informações em categorias mais amplas e que possibilitam a melhor análise geral da pesquisa relacionada. São dados amplos mas sensíveis para a compreensão do estado da ciência envolvida com os aspectos qualitativos do SOC.

% Em geral, o resultado de um estudo de mapeamento é um mapa visual classificando os resultados obtidos. Em particular, acreditamos que esse mapeamento pode beneficiar o estado da arte e da pr\'{a}tica em Computa\c{c}\~{a}o Orientada a Servi\c{c}os identificando tend\^{e}ncias e oportunidades para transfer\^{e}ncia de conhecimento. Considerando tamb\'{e}m o tamanho e abrang\^{e}ncia dessa \'{a}rea outros  objetivos do presente mapeamento de estudos s\~{a}o tamb\'{e}m esclarecer o paradigma da orientação a serviços no contexto de qualidade de servi\c{c}os (QoS) por meio de uma classificação ampla e sistemática, obtendo informações sobre frequências de publica\c{c}\~{o}es, \'{a}reas e t\'{o}picos de pesquisa, enfoques, tipos de contribui\c{c}\~{o}es de pesquisa dadas, os agentes e f\'{o}runs envolvidos.

% As demais se\c{c}\~{o}es desse artigo est\~{a}o organizadas da seguinte forma: Se\c{c}\~{a}o \ref{?}.Se\c{c}\~{a}o \ref{?}. Se\c{c}\~{a}o \ref{?}. Finalmente, na Se\c{c}\~{a}o \ref{?} apresentamos um resumo das nossas descobertas e provemos tamb\'{e}m algumas discuss\~{o}es finais. 

% %Neste mapeamento, deve-se prezar pelo uso de ferramentas de apoio e que agilizem os procedimentos sistemáticos a serem seguidos. Não é seu objetivo avaliar qualitativamente os trabalhos de pesquisa classificados, mantendo a análise a um nível menos detalhado e que permitirá a inclusão dessas publicações em categorias abrangentes e significativas e que não exijam a minuciosa análise de cada uma delas. 

% %Em relação aos atributos de QoS, tem-se por objetivo compreender que tipo de intervenções vem sendo feitas para a absorção de aspectos de qualidade em SOC e sua melhoria, identificando quais tipos de pesquisa,  de contribuição e quais atributos ou contextos tem maior importância, quais representam os maiores desafios para a concretização da adoção deste paradigma e quais são pouco abordados. Entre os variados e numerosos tipos de atributos que atendem a diversos modelos, este trabalho visa aqueles cuja definição encontra-se bem difundida e aceita entre os trabalhos relacionados. São eles o desempenho, a disponibilidade, a confiabilidade, a segurança, a modificabilidade, a testabilidade, a escalabilidade, o custo e outros para que os demais atributos sejam representados agrupadamente. Entre os tipos de pesquisa aceitos, estão os de avaliação, solução, validação, filosófico e de experiência pessoal. Por fim, entre os tipos de contribuições dadas, estão as de modelo, método, processo, ferramenta e métricas.

    
%TCIDATA{LaTeXparent=0 0 sbrc_2013.tex}
%Review Method

\section{Método do Estudo}\label{sec:review_method}

Este artigo apresenta os resultados de um mapeamento sistemático de estudos. Um MS tem por objetivo classificar informações acerca de uma área de pesquisa de forma ampla e menos minuciosa que a tradicional revisão sistemática de estudos. Uma vez constatada a vasta quantidade de publicações no campo de SOC, escolheu-se esta metodologia visando a viabilidade da tarefa de se classificar um número elevado de artigos. A metodologia adotada seguiu as diretrizes propostas em \cite{Petersen:2008:SMS:2227115.2227123}.

\subsection{Protocolo do Estudo}

Um mapeamento sistemático de estudos, assim como outras revisões literárias, estabelece o uso de um protocolo que irá documentar as etapas do mapeamento de modo a garantir sua replicação e diminuir possíveis erros por parte dos pesquisadores. Nele estão definidas as questões de pesquisa, os fóruns científicos onde as publicações serão buscadas, a \textit{string} de busca utilizada e os critérios de inclusão e exclusão de artigos. 

\subsection{Quest\~{o}es de Pesquisa}

As questões de pesquisa foram organizadas de acordo com a motivação desse estudo, que é investigar e categorizar as contribuições de pesquisa em computação orientada a serviço no contexto de qualidade de serviço. Esse estudo tem como objetivo responder às seguintes perguntas: 

\begin{itemize}
\item {\bf QP1} Qual o interesse de pesquisa da comunidade científica no tema nos \'{u}ltimos anos? 
\item {\bf QP2} Quais as áreas de SOC são mais frequentemente pesquisadas no contexto de qualidade de serviços?
\item {\bf QP3} Quais atributos de qualidade são frequentemente considerados nos estudos abordados?
\item {\bf QP4} Qual o foco da contribuição de pesquisa realizada?   
\end{itemize}

A QP1 almeja identificar o estado da pesquisa relacionada a SOC no contexto de qualidade de servi\c{c}os em termos quantitativos, ou seja, apontar o n\'{u}mero de contribui\c{c}\~{o}es por autor na \'{a}area e os principais autores envolvidos. A QP2 tem como objetivo trazer uma perspectiva do cen\'{a}rio das pesquisas em Computa\c{c}\~{a}o Orientada a Servi\c{c}o com foco em QoS atualmente. Para responder a essa pergunta, precisaremos delimitar primeiramente qual o per\'{i}odo que tenha uma representa\c{c}\~{a}o significativa a ser analisada, com base na frequ\^{e}ncia do n\'{u}mero de publica\c{c}\~{o}es principalmente nos \'{u}ltimos anos. Em segundo lugar, ainda com rela\c{c}\~{a}o \`{a} primeira quest\~{a}o de pesquisa, precisamos definir quais s\~{a}o as \'{a}reas que melhor caracterizam as diversas contribui\c{c}\~{o}es de pesquisa em SOC. Com rela\c{c}\~{a}o \`{a} QP3, pretendemos obter com esse estudo quais s\~{a}o os atributos de QoS mais frequentemente explorados em SOC. Em outras palavras, considerando que QoS, nesse contexto, envolve atributos como disponbilidade, confiabilidade, desempenho, seguran\c{c}a, escalabilidade, custo, SLA, quais desses atributos de QoS est\~{a}o de fato em foco. Por fim, a QP4 almeja elucidar quais tipos de pesquisa s\~{a}o mais frequentes e inferir conclusões acerca de sua maturidade. Vale ressaltar aqui, que n\~{a}o pretendemos avaliar o m\'{e}rito da pesquisa em si. Est\'{a} fora do escopo desse estudo avaliar tal quest\~{a}o.

% 	A QP1 almeja identificar o estado da pesquisa relacionada a SOC no contexto de qualidade de serviços em termos quantitativos, ou seja, apontar o número de publicações na área e os principais autores envolvidos. A QP2 busca mapear quais tópicos receberam maior atenção de pesquisa. Para isso, foram estabelecidas facetas de contribuição capazes de abranger relevantes atividades presentes no campo da computação orientada a serviços. A QP3 visa conhecer quais são, entre os atributos de qualidade de maior notoriedade, os que são com maior frequência contemplados. Além dos atributos de qualidade disponíveis, definiu-se dois outros itens de classificação de contexto. Um para representar a escolha genérica de atributos, isto é, contribuições que não definiram atributos específicos, e outro para representar atributos outros que não estejam definidos como itens de classificação de contexto. 

\subsection{Estratégia de Busca}\label{estrategia_busca}
%Our search strategy consisted of both manual and electronic search. Electronic search was performed in the following digital databases: ACM Digital Library, CiteSeerX, Compendex, Google Scholar, IEEE Xplore and SpringerLink. These are relevant electronic databases to computer science and software engineering, also used in a number of systematic studies in the area [2] .To formulate the search string for electronic database search, we used an approach suggested by Kitchenham [1]. The strategy derives the search string from the research questions using a composition with Boolean operators OR and AND. Table 1 presents the search string. The justification for using manual search was that creativity in RE is a relatively novel area, therefore manual search in conferences and journals provided extra confidence that relevant papers would be found.  To have a more representative set of studies, the “snow-balling" technique was adopted [3], in which the references of the identified papers were analyzed.
Nossa estratégia de busca consistiu essencialmente na busca eletrônica nas seguintes bibliotecas digitais: ACM Digital Library, ScienceDirect, IEEE Xplore e SpringerLink. Essas estão entre as bibliotecas mais relevantes para a Ciência da Computação. Para formular os termos de busca para a base de dados eletrônica, usamos a abordagem sugerida por Kitchenham\cite{}. A estrategia deriva os termos de busca a partir das questões de pesquisa  usando uma composição com os operadores OR e AND. Para evitar a tendenciosidade quanto a quais comunidades de pesquisa mais atuantes no contexto desse estudo, assim como obter um tamanho real do volume das contribuições, resolvemos não adotar técnicas como \emph{snow-balling} onde outros trabalhos relacionados podem ser encontrados a partir das referências dos trabalhos extraídos automaticamente \cite{}.
%VIDE Using Mapping Studies in Software Engineering (Proc. of PPIG 2008

\begin{table}[ht]
\centering
\caption{Termos de Busca utilizados para pesquisa de publicações}
\label{tab:exTable1}
\begin{tabular}{p{0.75\linewidth}}
\hline
((``web service'' OR ``web services'' OR ``service oriented'' OR ``service-oriented'' OR SOA OR SaaS OR PaaS OR ``service orientation'' OR ``service-oriented computing'' OR ``service oriented computing'' OR SOC) AND (``quality of services'' OR ``quality of service'' OR QOS)) \\
\hline
\end{tabular}
\end{table}


\subsection{Crit\'{e}rio de Inclusão e Exclusão}

Para filtrar os artigos coletados, utilizamos os seguintes critérios para inclus\~{a}o e exclus\~{a}o. Inclu\'{i}mos artigos publicados em workshops, confer\^{e}ncias e peri\'{o}dicos nas bibliotecas digitais, conforme descrito na Se\c{c}\~{a}o \ref{estrategia_busca}. Artigos considerados como \emph{grey literature}, i.e. relat\'{o}rios t\'{e}cnicos e \emph{white papers}, foram exclu\'{i}dos. No que tange \`{a}s contribui\c{c}\~{o}es em SOC, foram consideradas somente aquelas que se lidavam com n\'{i}veis de abstra\c{c}\~{a}o acima do sistema operacional, e.g. relativas a middleware ou plataformas de distribui\c{c}\~{a}o. 

\item Nenhum artigo cujo foco de contribuição esteja na camada de infraestrutura
\item No campo de SOC, somente pesquisas cuja contribuição esteja no contexto de qualidade de serviços
\item Apenas artigos com mais de 5 páginas
\item Somente artigos que proponham soluções, avaliações, validações e experiência pessoal do(s) autor(es), conforme definições em \cite{}.
\item Somente publicações com data posterior a 2008
\end{itemize}

\subsection{Seleção do Estudo}
%Falar sobre o \emph{crawler} (automatização da busca) em cada biblioteca digital e como foram armazenados em um site e configurados para a revisão colaborativa de todos os autores desse trabalho. Falar do n\'{u}mero total de artigos inicialmente e quantos foram filtrados ao final, em virtude do criterio de exclusão. Mencionar onde iremos disponbilizar a base dos artigos incluídos ao final.

Inicialmente foram feitas consultas manuais em cada uma das bibliotecas digitais mencionadas em \ref{estrategia_busca}. Verificou-se ao todo o número de 1034 publicações a serem analisadas. Para atender a esse número elevado, a coleta dos resultados de busca foi automatizada por um \emph{crawler} capaz de se comunicar com os sítios e persistir num banco de dados local os metadados das publicações resultantes das buscas nas diferentes bibliotecas.

\subsection{Extração de Dados e Análise}

%Descrever como o ambiente no Heroku está organizado (facets), explicar os termos principalmente os de Computação Orientada a Serviços qual a referência de significado que usamos. ``Each author individually extracted data from a subset of papers. We jointly  discussed unclear issues and solved discrepancies in the analysis.'' Os resultados tambem foram gerados automaticamente por meio da propria ferramenta....
Dada a dificuldade de se trabalhar cooperativamente com planilhas digitais e visando uma maior eficiência e ubiquidade de trabalho, desenvolveu-se uma ferramenta de apoio capaz de facilitar a extração de dados e gerar resultados em tempo real. Esta consiste num ambiente disponível em nuvem, com interfaces disponíveis para a listagem das publicações coletadas automaticamente pelo do \emph{crawler} ou de forma manual pela interface de registro de novas publicações.

No ambiente dessa ferramenta, cada publicação pode ser classificada por meio de interface apropriada que contém os metadados do artigo, campos de anotações e marcações dos itens de classificação definidos para o mapeamento de estudos em questão. O uso dessa ferramenta foi de grande importância para a viabilidade do MS diante da quantidade inicial de publicações coletadas. Al\'{e}m disso, tal ferramenta permite a realiza\c{c}\~{a}o do mapeamento de forma colaborativa onde os artigos s\~{a}o compartilhados entre diferentes grupos de usuários em suas respectivas sess\~{o}es autenticadas.

A partir da distribuição automática de artigos para cada um dos pesquisadores, foram excluídas manualmente as publicações que não se adequaram aos critérios definidos. Tal processo, assim como a classificação, foi realizado individualmente, tendo havido frequente discussão para eliminar quaisquer dúvidas e inconsistências de interpretações quanto aos critérios de inclusão, exclusão e facetas de classificação. 

Os artigos foram classificados de acordo com as categorias:
\begin{itemize}
\item[-] \textbf{Contribuição}: composição, coordenação e comunicação, descoberta e seleção, ciclo de vida, monitoramento e adaptação, modelos de QoS e linguagens. As definições usadas para as facetas de contribuição são definidas conforme~\cite{SCUBE}. Em particular, a categoria \emph{modelos de QoS e linguagens} engloba publicações que definem extensões ou novos modelos de QoS por meio de linguagens, especificações e ontologias a serem utilizadas em sistemas baseados em serviços e na definição de contratos de serviços que incluam garantias de QoS
\item[-] \textbf{Contexto}: disponibilidade, desempenho, confiabilidade, escalabilidade, segurança, SLA, custo, outros)... (EXPLICAR EM PARTICULAR SLA). 
\item[-] \textbf{Pesquisa}: solução, avaliação, validação e experiência pessoal. Em particular, diferenciamos a caracter\'{i}stica \emph{avalia\c{c}\~{a}o} de \emph{valida\c{c}\~{a}o} conforme \cite{Wiering et al. 2006}. Artigos classificados em avalia\c{c}\~{a}o s\~{a}o aqueles que .... Enquanto que os artigos avaliados em valida\c{c}\~{a}o s\~{a}o aqueles que .... No entanto, vale ressaltar que artigos classificados como solu\c{c}\~{a}o podem tamb\'{e}m podem ser mapeados como valida\c{c}\~{a}o ou avalia\c{c}\~{a}o, caso estas fa\c{c}am parte da contribui\c{c}\~{a}o do trabalho estudado.
\end{itemize}

%\begin{itemize}
%\item {\bf Composição} Trata da agregação de serviços visando o estabelecimento de novas aplicações ou seu rearranjo de forma a manter determinados níveis de QoS pré-estabelecidos ou acordados.
%\item {\bf Coordenação e Comunicação} Trata da orquestração de serviços para atender a um objetivo determinado num período de duração correspondente à atividade executada. Além disso, envolve a troca de mensagens através do orquestrador, portanto também envolve protocolos específicos para a transação e comunicação em geral.
%\item {\bf Descoberta e Seleção} Refere-se à publicação, descoberta e seleção de serviços em registros públicos ou privados considerando aspectos de qualidade de serviços.
%\item {\bf Ciclo de vida} Refere-se às fases da engenharia de software dentro do domínio de sistemas baseados em serviços, tal qual projeto, desenvolvimento, manutenção e testes.
%\item {\bf Monitoramento e Adaptação} Relacionados ao tempo de execução, monitoramento de atributos não funcionais e adaptação de configuração e disposição de serviços visando manter os níveis desejados de QoS.
%\item {\bf Modelos de QoS e Linguagens} Engloba publicações que definem extensões ou novos modelos de QoS por meio de linguagens, especificações e ontologias a serem utilizadas em sistemas baseados em serviços e na definição de contratos de serviços que incluam garantias de QoS.
%\end{itemize} 

\section{Resultados}


\subsection{Quest\~{a}o de Pesquisa 1}
\emph{Qual o interesse de pesquisa da comunidade científica no tema nos \'{u}ltimos anos?}
\section{Questão de Pesquisa 2}

\emph{Quais as áreas de SOC são mais frequentemente pesquisadas no contexto de qualidade de serviços?}


\subsection{Questão de Pesquisa 3}\label{sec:QP3}

\emph{Quais s\~{a}o os estudos existentes que mais tem impulsionado QoS em SOC?}

%Com essa quest\~{a}o de pesquisa, pretendemos tamb\'{e}m obter quais s\~{a}o os autores com maior n\'{u}mero de publica\c{c}\~{o}es relacionadas ao contexto desse estudo.

Identificamos quatro principais grupos de pesquisa que mais contribu\'{i}ram com pesquisas em SOC no contexto de QoS. Classificamos esses grupos conforme segue: 

\textbf{Grupo S-Cube} -- Identificamos que dos quatros grupos que mais contribu\'{i}ram no \^{a}mbito desse estudo est\~{a}o pesquisadores cuja afilia\c{c}\~{a}o est\'{a} inserida direta ou indiretamente no contexto do grupo europeu S-Cube\~cite{SCube}. No per\'{i}odo do nosso estudo, o grupo S-Cube contribuiu com 10 publica\c{c}\~{o}es relevantes para esse contexto. S\~{a}o os seguintes autores Schahram Dustdar (Vienna University of Techonology) e Raffaela Mirandola (Politecnico di Milano) com seus colaboradores Valeria Cardelini e Emiliano Casalicchio ambos de Universit\'{a} di Roma ``Tor Vergata''. Em particular, percebemos que Mirandola e seus colaboradores mais tem contribu\'{i}do em pesquisas relacionadas monitoramento e adapta\c{c}\~{a}o no contexto de confiabilidade, disponibilidade e desempenho como pode ser percebido com as publica\c{c}\~{o}es \cite{Cardellini:2009:QRA:1595696.1595718, Calinescu:2011:DQM:1990772.1991012, Ardagna:2010:POS:1814581.1814611, 10.1109/TSE.2011.68, Cardellini:2009:TSD:1692867.1692870}. Dustdar e seus colaboradores, por sua vez, tem mais contribu\'{i}do em composi\c{c}\~{a}o de servi\c{c}os em ambientes din\^{a}micos no escopo de SLA, com destaque para o  VRESCO (\emph{Vienna Runtime Environment for Service-Oriented Computing})~\cite{5467022}. Em particular, segundo a base da IEEE Xplore mostra que essa contribui\c{c}\~{a}o recebeu, at\'{e} o momento desse estudo, em torno de 42 cita\c{c}\~{o}es desde Setembro de 2010.

\textbf{Daniel Menasce et al.} -- Menasc\'{e} e seus colaboradores t\^{e}m tradicionalmente contribu\'{i}do com pesquisas relativas a QoS, em particular no \^{a}mbito de desempenho, incluindo os diversos f\'{o}runs da \'{a}rea~\cite{DBLP:journals/internet/Menasce04b, DBLP:journals/pe/MenasceARRFM03, DBLP:journals/tse/MenasceG00, Menasce:2001:CPW:560806}. 

No per\'{i}odo do nosso estudo, Menasc\'{e} et al. contribu\'{i}ram com 6 publica\c{c}\~{o}es e destacaram-se no contexto de SLA nas \'{a}reas de descobrimento \& sele\c{c}\~{a}o e monitoramento \& adapta\c{c}\~{a}o~\cite{5696721, DBLP:MenasceCD10, 5552741}. Em~\cite{5696721}, Menasc\'{e} et al. contribuem com o SASSY, um arcabou\c{c}o que gera automaticamente arquiteturas de software candidatas e seleciona aquela que melhor se adequa ao objetivo de QoS. Em~\cite{DBLP:MenasceCD10}, eles apresentam um algoritmo que encontra a solu\c{c}\~{a}o para otimiza\c{c}\~{a}o na busca de provedores de servi\c{c}o com restri\c{c}\~{o}es de custo e tempo de execu\c{c}\~{a}o. Vale ressaltar que esse trabalho teve 39 cita\c{c}\~{o}es at\'{e} o momento desse estudo e foi realizado em coopera\c{c}\~{a}o com Casalicchio, que tamb\'{e}m teve colabora\c{c}\~{o}es com o grupo do S-Cube.  

\textbf{Kwei-Jay Lin et al.} -- O grupo de Lin (University of California, Irvine) tem contribu\'{i}do essencialmente no contexto de SLA no \^{a}mbito de composi\c{c}\~{a}o (din\^{a}mica) e adapta\c{c}\~{a}o de servi\c{c}os. Particular destaque para  a contribui\c{c}\~{a}o \cite{Lin:2009:EAS:1602932.1603224, Lin:2010:DIS:1861294.1861332, Zhai:2009:SMS:1586636.1586972} que lida com reconfigura\c{c}\~{a}o de servi\c{c}os em SOA~\cite{SOA REFERENCE}, com restri\c{c}\~{o}es de QoS fim-a-fim. 

No Brasil em particular, o grupo que mais tem contrbu\'{i}do com o tema no per\'{i}odo do estudo:

\textbf{Ricardo Rabelo et al.} --  Rabelo e seu grupo tem contribu\'{i}do com mecanismos de descoberta de servi\c{c}o no contexto de QoS tanto para alavancar servic\c{c}os de software compartilhados sobre o que chamam de redes colaborativas~\cite{conf/ifip5-5/Perin-SouzaR11} quanto para integrar BPM e SOA no contexto de provedores de servi\c{c}o de software largamente distribu\'{i}dos~\cite{Perin-Souza:2010:AMA:1909623.1909668}.

Observamos que caso tiv\'{e}ssemos definido termos de QoS em particular, como dependabilidade ou desempenho no termo de busca, outros grupos poder\'{i}am ter-se destacado no \^{a}mbito brasileiro, como \'{e} o caso do grupo de Rubira et al. (Unicamp). Mas por quest\~{a}o de viabilidade do estudo, restringimo-nos ao termo de qualidade de servi\c{c}o.
\subsection{Questão de Pesquisa 4}

\emph{Qual o foco da contribuição de pesquisa realizada em SOC e relacionada com qualidade de serviço? }

Com o intuito de responder a esta questão, foi feita uma avalia\c c\~{a}o da distribui\c c\~{a}o dos 
artigos em rela\c c\~{a}o ao tipo de pesquisa (conforme discutido na Se\c c\~{a}o~\ref{sec:review_method}). 
Os \emph{bubble plots} nas Figuras~\ref{} e~\ref{}  apresentam tal distibui\c c\~{a}o, novamente sendo importante 
ressaltar que o n\'{u}mero total de artigos nos gr\'{a}ficos \'{e} superior ao n\'{u}mero total de artigos analisados--- 
uma vez que alguns artigos apresentam contribui\c c\~{o}es tanto em termos de uma nova solu\c c\~{a}o proposta 
quanto em termos de avalia\c c\~{a}o e/ou valida\c c\~{a}o. Por exemplo, Huang et al. prop\~{o}e um 
modelo estoc\'{a}stico para representar e raciocinar sobre dependabilidade em um ambiente de SOC, ao mesmo tempo 
que valida formalmente tal proposta por meio de provas de teoremas~\cite{huang:scc2011}.

Esta investiga\c c\~{a}o revelou que 21 artigos (como por exemplo~\cite{jeong:fqs2009,ardagna:jss2010}) 
contribuem com uma nova solu\c c\~{a}o para 
lidar com qualidade de servi\c co em SOC ao mesmo tempo que apresentam uma investiga\c c\~{a}o tanto 
em termos de avalia\c c\~{a}o quanto em termos de valida\c c\~{a}o. Al\'{e}m disso, 80 
artigos s\~{a}o propostas de solu\c c\~{o}es que apresentam 
avalia\c c\~{o}es em termos de estudos de casos mais simples, n\~{a}o compreendendo uma 
valida\c c\~{a}o da(s) t\'{e}cnica(s) proposta(s). Entre esses artigos podemos 
citar~\cite{filieri:faa2012, pernici:services2011,nascimento:splc2011}. Finalmente, 27 artigos apresentam, 
al\'{e}m de uma nova solu\c c\~{a}o relacionada \`{a} qualidade de servi\c cos em SOC, 
uma s\'{o}lida valida\c c\~{a}o (e.g.~\cite{huang:scc2011,binshtok:icsoc2009}).    
Por outro lado, 102 artigos (aproximadamente 40\% do total) prop\~{o}em novas 
solu\c c\~{o}es sem apresentar avalia\c c\~{a}o ou valida\c c\~{a}o consistentes (\cite{balfagih:icime2011,fiadeiro:fac2011,khazankin:scc2011}). 
Enquanto que apenas 4 artigos focam na avalia\c c\~{a}o e/ou valida\c c\~{a}o de 
propostas existentes~\cite{voelz:edoc2010,moayed:icsea2010,cavallo:pesos2010,banerjee:isorcw2011}.  

Esses n\'{u}meros revelam que a \'{a}rea de pesquisa de qualidade de servi\c co em 
SOC ainda est\'{a} em uma fase 
de amadurecimento, onde um percentual significativo das contribui\c c\~{o}es 
simplesmente apresentam novas abordagens ou fazem compara\c c\~{o}es envolvendo a 
pr\'{o}pria t\'{e}cnica proposta--- o que pode levar a conclus\~{o}es tendenciosas. A quantidade 
de artigos que visam avaliar ou validar t\'{e}cnicas existentes, algo recomendado antes 
de se iniciar a concep\c c\~{a}o de uma nova solu\c c\~{a}o, \'{e} praticamente insignificante. 
 

\section{Questão de Pesquisa 5}

\emph{Qual o foco da contribuição de pesquisa realizada em SoC e relacionada com qualidade de serviço? }

Com o intuito de responder a esta questão, foi feita uma avalia\c c\~{a}o da distribui\c c\~{a}o dos 
artigos em rela\c c\~{a}o ao tipo de pesquisa (conforme discutido na Se\c c\~{a}o~\ref{sec:review_method}). 
Os \emph{bubble plots} nas Figuras~\ref{} e~\ref{}  apresentam tal distibui\c c\~{a}o, novamente sendo importante 
ressaltar que o n\'{u}mero total de artigos nos gr\'{a}ficos \'{e} superior ao n\'{u}mero total de artigos analisados--- 
uma vez que alguns artigos apresentam contribui\c c\~{o}es tanto em termos de uma nova solu\c c\~{a}o proposta 
quanto em termos de avalia\c c\~{a}o e/ou valida\c c\~{a}o. Por exemplo, Huang et al. prop\~{o}e um 
modelo estoc\'{a}stico para representar e raciocinar sobre dependabilidade em um ambiente de SoC, ao mesmo tempo 
que valida formalmente tal proposta por meio de provas de teoremas~\cite{huang:scc2011}.

Esta investiga\c c\~{a}o revelou que 21 artigos (como por exemplo~\cite{jeong:fqs2009,ardagna:jss2010}) 
contribuem com uma nova solu\c c\~{a}o para 
lidar com qualidade de servi\c co em SoC ao mesmo tempo que apresentam uma investiga\c c\~{a}o tanto 
em termos de avalia\c c\~{a}o quanto em termos de valida\c c\~{a}o. Al\'{e}m disso, 80 
artigos s\~{a}o propostas de solu\c c\~{o}es que apresentam 
avalia\c c\~{o}es em termos de estudos de casos mais simples, n\~{a}o compreendendo uma 
valida\c c\~{a}o da(s) t\'{e}cnica(s) proposta(s). Entre esses artigos podemos 
citar~\cite{filieri:faa2012, pernici:services2011,nascimento:splc2011}. Finalmente, 27 artigos apresentam, 
al\'{e}m de uma nova solu\c c\~{a}o relacionada \`{a} qualidade de servi\c cos em SoC, 
uma s\'{o}lida valida\c c\~{a}o (e.g.~\cite{huang:scc2011,binshtok:icsoc2009}).    
Por outro lado, 102 artigos (aproximadamente 40\% do total) prop\~{o}em novas 
solu\c c\~{o}es sem apresentar avalia\c c\~{a}o ou valida\c c\~{a}o consistentes (\cite{balfagih:icime2011,fiadeiro:fac2011,khazankin:scc2011}). 
Enquanto que apenas 4 artigos focam na avalia\c c\~{a}o e/ou valida\c c\~{a}o de 
propostas existentes~\cite{voelz:edoc2010,moayed:icsea2010,cavallo:pesos2010,banerjee:isorcw2011}.  

Esses n\'{u}meros revelam que a \'{a}rea de pesquisa de qualidade de servi\c co em 
SoC ainda est\'{a} em uma fase 
de amadurecimento, onde um percentual significativo das contribui\c c\~{o}es 
simplesmente apresentam novas abordagens ou fazem compara\c c\~{o}es envolvendo a 
pr\'{o}pria t\'{e}cnica proposta--- o que pode levar a conclus\~{o}es tendenciosas. A quantidade 
de artigos que visam avaliar ou validar t\'{e}cnicas existentes, algo recomendado antes 
de se iniciar a concep\c c\~{a}o de uma nova solu\c c\~{a}o, \'{e} praticamente insignificante. 
 





\section{Amea\c cas a validade}\label{sec:ameacas}

As poss\'{i}veis amea\c{c}as a aos resultados dessa pesquisa, bem como a generaliza\c c\~{a} dos mesmos, s\~{a}o descritas nessa se\c c\~{a}o, conforme a classifica\c c\~{a}o apresentada em~\cite{leedy1980practical}. 

\noindent
\emph{A Validade de constru\c c\~{a}o} est\'{a} relacionada
\`{a}s decis\~{o}es operacionais que foram tomadas, durante o
planejamento do estudo, com o intuito de responder \`{a}s quest\~{o}es de
pesquisa. As principais constru\c{c}\~{o}es do nosso estudo est\~{a}o em definir os conceitos de ``QoS em SOC'', definir o per\'{i}odo de publica\c{c}\~{o}es significativa para o estudo e \textbf{o mapeamento sistemático em si}. Conceitos de SOC:


Como nosso objetivo \'{e} identificar tend\^{e}ncias de pesquisa na \'{a}rea de QoS em SOC, o fato de n\~{a}o termos considerado trabalhos publicados ap\'{o}s Dezembro de 2011 pode representar uma amea\c ca aos nossos resultados. Por outro lado, consideramos que trabalhos publicados entre Janeiro de 2009 e Dezembro de 2011 permitem-nos avaliar um per\'{i}odo bastante significativo do estudo, conforme ilustrado na Figura~\ref{fig:barplotAnoPublicacoes}. Al\'{e}m disso, consideramos que uma avalia\c c\~{a}o parcial do ano corrente (2012) poderia levar a avalia\c{c}\~{o}es inconclusivas quanto \`{a} tend\^{e}ncia do estudo. Por fim, quanto \`{a} terceira amea\c{c}a de constru\c{c}\~{a}o, para realizar nosso estudos seguimos as diretrizes de ref\^{e}ncia de Kitchenham~\cite{kitchenham:techReport2007} para definir as quest\~{o}es de pesquisa, crit\'{e}rio de busca e protocolo do estudo.

\noindent
\emph{A Validade interna}
Estabelece a rela\c{c}\~{a}o causal garantindo que certas condi\c{c}\~{o}es levem de fato a outras condi\c{c}\~{o}es. Em particular, a ameaça nesse sentido consiste na seleção incompleta ou inadequada das contribuições ou a tendenciosidade da visão individual de cada pesquisador envolvido. Um estudo abrangente como este, envolvendo cinco pesquisadores, leva a algum risco nesse sentido, uma vez que a classifica\c c\~{a}o dos artigos pode envolver certo grau subjetividade na interpreta\c{c}\~{a}o de cada faceta. Com o intuito de minimizar tal amea\c{c}a, algumas reuni\~{o}es objetivaram esclarecer as facetas do estudo, enquanto que outras serviram para realizar avalia\c c\~{o}es em pares. Foram feitas diversas discuss\~{o}es entre os autores deste trabalho sempre que a an\'{a}lise de um artigo resultava em d\'{u}vidas quanto sua classifica\c c\~{a}o. Finalmente, a seleção correta pode ser percebida tanto definição adequada do protocolo na Seção~\ref{sec:review_method} quanto por resultados conclusivos como aqueles apresentados por principais grupos na Se\c{c}\~{a}o ~\ref{sec:QP3}.

\noindent
\emph{A Validade externa} est\'{a} relacionada com a possibilidade de
generaliza\c c\~{a}o do nosso estudo. Esse \'{e} um aspecto muito
importante que est\'{a} relacionado ao escopo do nosso estudo, que tem
como objetivo entender a pesquisa realizada em QoS / SOC. Por outro
lado, o termo QoS \'{e} bastante amplo, sendo invi\'{a}vel realizar
um estudo compreensivo sobre todos os atributos de qualidade
relacionados a SOC (seguran\c ca, performance, toler\^{a}ncia a falhas
etc.). Nesse sentido, e conforme discutido na se\c{c}\~{a}o~\ref{sec:review_method}, a nossa \texttt{string} de busca se restringiu aos termos \emph{quality of services}, \emph{quality of service} e \emph{QOS}. Tais termos deveriam aparecer em algum campo de metadados de um artigo (como t\'{i}tulo, resumo ou palavras-chave) para que o mesmo fosse recuperado pelo \emph{crawler} desenvolvido. Ou seja, n\~{a}o generalizamos nossos resultados para pesquisas desenvolvidas em uma \'{a}rea espec\'{i}fica de QOS, mas sim para a \'{a}rea de QOS de forma mais gen\'{e}rica. Como
trabalho futuro, temos o interesse de reproduzir mapeamentos sistem\'{a}ticos de estudo para algumas \'{a}reas espec\'{i}ficas de QOS (tratamento de exce\c
c\~{o}es, seguran\c ca etc). 

\noindent
\emph{A confiabilidade} est\'{a} relacionada ao grau de objetividade de
um estudo, em que uma poss\'{i}vel replica\c c\~{a}o levaria a
resultados parecidos. Com a infraestrutura que disponibilizamos para a realiza\c{c}\~{a}o de mapeamentos sistem\'{a}ticos, podemos tanto estender essa investiga\c c\~{a}o em um trabalho futuro quanto convidarmos outros pesquisadores para replicar tal avalia\c c\~{a}o. Em particular, a estrat\'{e}gia de s\'{i}ntese e interpreta\c{c}\~{a}o escolhidas podem trazer diferentes \emph{insights}, mas acreditamos que as tend\^{e}ncias observadas devem permanecer as mesmas.

\section{Conclus\~{a}o}\label{sec:conclusao}

\subsection{Threats to Validity}

\subsection{Discussion}


\bibliographystyle{sbc}
\bibliography{sbc-template}

\end{document}
