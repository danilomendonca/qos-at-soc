\section{Conclus\~{a}o e Trabalho Futuro}\label{sec:conclusao}

Esse artigo apresenta um mapeamento sistem\'{a}tico na \'{a}rea 
de \emph{Qualidade de Servi\c co} em \emph{Computa\c c\~{a}o Orientada a Servi\c cos}, visando identificar 
(a) o tipo e maturidade da pesquisa realizada; (b) os atributos de qualidade e componentes de SOC considerados; e (c) os principais grupos interessados no tema. Os dados obtidos tra\c{c}aram um panorama da pesquisa relacionada, servindo de insumo e orientação para o aprofundamento de novas pesquisas nas diferentes tópicos envolvidos e identificando aspectos importantes da maturidade da pesquisa dentro do período avaliado. Consideramos que a definição dos tópicos mais relevantes de QoS em SOC como uma faceta de classificação, constituiu também uma das contribuições do artigo ao final do estudo. Além disso, determinou-se quais aspectos de qualidade tiveram maior destaque entre as pesquisas avaliadas, evidenciando não só o interesse dos pesquisadores, mas também possíveis lacunas de pesquisa em termos de QoS em ambientes orientados a serviços. Por fim, foi possível a identificação de grupos de pesquisa mais ativos na área em questão, o que possibilita o aprofundamento dos resultados por meio de técnicas de \textit{snow-balling} e também fornece um ponto de partida para revisões sistemáticas de literatura.

Como trabalho futuro, consideramos aprimorar a ferramenta de mapeamento criada para viabilizar esse estudo de modo a permitir melhor usabilidade com as bibliotecas digitais e viabilizar o uso de técnicas como \emph{snowball}. Quanto ao mapeamento sistem\'{a}tico em si, pretendemos utilizar a base de dados aceita para a realiza\c{c}\~{a}o de uma revis\~{a}o sistem\'{a}tica da literatura (\emph{Systematic Literature Review}) com vistas a uma avalia\c{c}\~{a}o mais aprofundada dos trabalhos avaliados, como por exemplo, seus benef\'{i}cios e limita\c{c}\~{o}es, assim como suas contribui\c{c}\~{o}es relativas \`{a} arquitetura de software.

%Entre as principais considera\c c\~{o}es, podemos destacar que a pesquisa envolvendo QoS e SOC parece estar nos est\'{a}gios iniciais, uma vez que boa parte do esfor\c co est\'{a} relacionada a proposi\c c\~{a}o de novas t\'{e}cnicas--- sendo que poucos trabalhos buscam coletar evid\^{e}ncias da efic\'{a}cia de propostas existentes.  Al\'{e}m disso,  

%Finalmente, \'{e} importante observar que alguns estudos apontam para uma promissora intera\c{c}\~{a}o entre nuvens computacionais e SOC~\cite{10.1109/MIC.2010.147}, onde a maior intersec\c{c}\~{a}o entre ambas seria no contexto de \emph{software} como serviço. Para atender a essa nova atua\c{c}\~{a}o para SOC, acreditamos que muitos pontos levantados nesse mapeamento podem ser relevantes para o modelo de computa\c{c}\~{a}o em larga escala proposto pela computação em nuvens, o que poderá ser verificado num trabalho futuro.