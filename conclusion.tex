\section{Conclus\~{a}o}\label{sec:conclusao}

Esse artigo apresenta um mapeamento sistem\'{a}tico nas \'{a}rea 
de \emph{Qualidade de Servi\c co} em \emph{Computa\c c\~{a}o Orientada a Servi\c cos}, visando identificar 
(a) o tipo e maturidade da pesquisa realizada; (b) os atributos de qualidade e componentes de SOC considerados; 
e (d) os principais grupos interessados no tema. Entre as principais considera\c c\~{o}es, podemos destacar que 
a pesquisa envolvendo QoS e SOC parece estar nos est\'{a}gios iniciais, uma vez que boa parte do esfor\c co est\'{a} 
relacionada a proposi\c c\~{a}o de novas t\'{e}cnicas--- sendo que poucos trabalhos buscam coletar evid\^{e}ncias 
da efic\'{a}cia de propostas existentes.  Al\'{e}m disso, observamos que a maior parte dos trabalhos lidam com 
os componentes de monitoramento e adapta\c c\~{a}o, al\'{e}m de propostas de novos modelos de QoS e linguagens para 
especificar atributos de QoS. 
Finalmente, \'{e} importante observar que alguns estudos apontam para uma promissora intera\c{c}\~{a}o entre nuvens computacionais e SOC~\cite{10.1109/MIC.2010.147}, onde a maior intersec\c{c}\~{a}o entre ambas seria no contexto de \emph{software} como serviço. Para atender a essa nova atua\c{c}\~{a}o para SOC, acreditamos que muitos pontos levantados nesse mapeamento podem ser relevantes para o modelo de computa\c{c}\~{a}o em larga escala proposto pela computação em nuvens, o que poderá ser verificado num trabalho futuro.