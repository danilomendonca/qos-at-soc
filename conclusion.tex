\section{Conclus\~{a}o}\label{sec:conclusao}

Com base nos resultados evidenciados por esse estudo sistem\'{a}tico, \'{e} poss\'{i}vel confirmar v\'{a}rios aspectos relevantes 

O resultado desse mapeamento mostra que a maior parte dos trabalhos publicados lida com monitoramento e adapta\c{c}\~{a}o (34.94\%), seguida de modelos de QoS \& linguagens (33.73\%), descoberta \& sele\c{c}\~{a}o (28.11\%), composi\c{c}\~{a}o (27.31\%). Esses resultados indicam o foco dado a aspectos n\~{a}o funcionais, din\^{a}micos e que podem sofrer varia\c{c}\~{o}es devido a concorrência e possíveis falhas dos servi\c{c}os em tempo de execu\c{c}\~{a}o. Mas tamb\'{e}m indicam o foco em propostas que visam a representa\c{c}\~{a}o e a escolha da melhor configura\c{c}\~{a}o de modo a atender aos níveis globais desej\'{a}veis ou necess\'{a}rios de QoS para um conjunto de servi\c{c}os.

No contexto de SOC, os dados gerais nos induzem a concluir que desempenho, disponibilidade e confiabilidade s\~{a}o prioridade como atributos de QoS em SOC. No entanto, o mesmo n\~{a}o pode ser concluído, com base nesse estudo, para seguran\c{c}a, escalabilidade e custo.

Quanto aos grupos que mais contribu\'{i}ram para o contexto de mapeamento est\~{a}o 3 principais grupos: (1) o S-Cube (com foco em monitoramento \& adapta\c{c}\~{a}o, assim como composi\c{c}\~{a}o din\^{a}mica), (2) Menasc\'{e} et al. (com foco em descobrimento \& sele\c{c}\~{a}o, assim como monitoramento \& adapta\c{c}\~{a}o) e (3) Lin et al (composi\c{c}\~{a}o din\^{a}mica e adapta\c{c}\~{a}o). Percebemos tamb\'{e}m que suas contribui\c{c}\~{o}es se encaixam na \'{a}reas de SOC que mais de destacaram em nosso estudo. 

No que tange \`{a} \'{a}rea de pesquisa de qualidade de servi\c{c}o em SOC, ainda est\'{a} em uma fase de amadurecimento, onde um percentual significativo das contribui\c{c}\~{o}es simplesmente apresentam novas abordagens ou fazem compara\c{c}\~{o}es envolvendo a própria t\'{e}cnica proposta -- o que pode levar a conclus\~{o}es tendenciosas. A quantidade de artigos que visam avaliar ou validar t\'{e}cnicas existentes, algo recomendado antes de se iniciar a concep\c{c}\~{a}o de uma nova solu\c{c}\~{a}o, \'{e} praticamente insignificante.