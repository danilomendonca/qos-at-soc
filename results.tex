\section{Resultados}\label{sec:resultados}

Nesta se\c c\~{a}o apresentamos os principais resultados do nosso
estudo, agrupados pelas quest\~{o}es de pesquisa discutidas na se\c
c\~{a}o anterior. 


\subsection{Quest\~{a}o de Pesquisa 1}
\emph{Qual o interesse de pesquisa da comunidade científica no tema nos \'{u}ltimos anos?}
\section{Questão de Pesquisa 2}

\emph{Quais as áreas de SOC são mais frequentemente pesquisadas no contexto de qualidade de serviços?}


\subsection{Questão de Pesquisa 3}\label{sec:QP3}

\emph{Quais s\~{a}o os estudos existentes que mais tem impulsionado QoS em SOC?}

%Com essa quest\~{a}o de pesquisa, pretendemos tamb\'{e}m obter quais s\~{a}o os autores com maior n\'{u}mero de publica\c{c}\~{o}es relacionadas ao contexto desse estudo.

Identificamos quatro principais grupos de pesquisa que mais contribu\'{i}ram com pesquisas em SOC no contexto de QoS. Classificamos esses grupos conforme segue: 

\textbf{Grupo S-Cube} -- Identificamos que dos quatros grupos que mais contribu\'{i}ram no \^{a}mbito desse estudo est\~{a}o pesquisadores cuja afilia\c{c}\~{a}o est\'{a} inserida direta ou indiretamente no contexto do grupo europeu S-Cube\~cite{SCube}. No per\'{i}odo do nosso estudo, o grupo S-Cube contribuiu com 10 publica\c{c}\~{o}es relevantes para esse contexto. S\~{a}o os seguintes autores Schahram Dustdar (Vienna University of Techonology) e Raffaela Mirandola (Politecnico di Milano) com seus colaboradores Valeria Cardelini e Emiliano Casalicchio ambos de Universit\'{a} di Roma ``Tor Vergata''. Em particular, percebemos que Mirandola e seus colaboradores mais tem contribu\'{i}do em pesquisas relacionadas monitoramento e adapta\c{c}\~{a}o no contexto de confiabilidade, disponibilidade e desempenho como pode ser percebido com as publica\c{c}\~{o}es \cite{Cardellini:2009:QRA:1595696.1595718, Calinescu:2011:DQM:1990772.1991012, Ardagna:2010:POS:1814581.1814611, 10.1109/TSE.2011.68, Cardellini:2009:TSD:1692867.1692870}. Dustdar e seus colaboradores, por sua vez, tem mais contribu\'{i}do em composi\c{c}\~{a}o de servi\c{c}os em ambientes din\^{a}micos no escopo de SLA, com destaque para o  VRESCO (\emph{Vienna Runtime Environment for Service-Oriented Computing})~\cite{5467022}. Em particular, segundo a base da IEEE Xplore mostra que essa contribui\c{c}\~{a}o recebeu, at\'{e} o momento desse estudo, em torno de 42 cita\c{c}\~{o}es desde Setembro de 2010.

\textbf{Daniel Menasce et al.} -- Menasc\'{e} e seus colaboradores t\^{e}m tradicionalmente contribu\'{i}do com pesquisas relativas a QoS, em particular no \^{a}mbito de desempenho, incluindo os diversos f\'{o}runs da \'{a}rea~\cite{DBLP:journals/internet/Menasce04b, DBLP:journals/pe/MenasceARRFM03, DBLP:journals/tse/MenasceG00, Menasce:2001:CPW:560806}. 

No per\'{i}odo do nosso estudo, Menasc\'{e} et al. contribu\'{i}ram com 6 publica\c{c}\~{o}es e destacaram-se no contexto de SLA nas \'{a}reas de descobrimento \& sele\c{c}\~{a}o e monitoramento \& adapta\c{c}\~{a}o~\cite{5696721, DBLP:MenasceCD10, 5552741}. Em~\cite{5696721}, Menasc\'{e} et al. contribuem com o SASSY, um arcabou\c{c}o que gera automaticamente arquiteturas de software candidatas e seleciona aquela que melhor se adequa ao objetivo de QoS. Em~\cite{DBLP:MenasceCD10}, eles apresentam um algoritmo que encontra a solu\c{c}\~{a}o para otimiza\c{c}\~{a}o na busca de provedores de servi\c{c}o com restri\c{c}\~{o}es de custo e tempo de execu\c{c}\~{a}o. Vale ressaltar que esse trabalho teve 39 cita\c{c}\~{o}es at\'{e} o momento desse estudo e foi realizado em coopera\c{c}\~{a}o com Casalicchio, que tamb\'{e}m teve colabora\c{c}\~{o}es com o grupo do S-Cube.  

\textbf{Kwei-Jay Lin et al.} -- O grupo de Lin (University of California, Irvine) tem contribu\'{i}do essencialmente no contexto de SLA no \^{a}mbito de composi\c{c}\~{a}o (din\^{a}mica) e adapta\c{c}\~{a}o de servi\c{c}os. Particular destaque para  a contribui\c{c}\~{a}o \cite{Lin:2009:EAS:1602932.1603224, Lin:2010:DIS:1861294.1861332, Zhai:2009:SMS:1586636.1586972} que lida com reconfigura\c{c}\~{a}o de servi\c{c}os em SOA~\cite{SOA REFERENCE}, com restri\c{c}\~{o}es de QoS fim-a-fim. 

No Brasil em particular, o grupo que mais tem contrbu\'{i}do com o tema no per\'{i}odo do estudo:

\textbf{Ricardo Rabelo et al.} --  Rabelo e seu grupo tem contribu\'{i}do com mecanismos de descoberta de servi\c{c}o no contexto de QoS tanto para alavancar servic\c{c}os de software compartilhados sobre o que chamam de redes colaborativas~\cite{conf/ifip5-5/Perin-SouzaR11} quanto para integrar BPM e SOA no contexto de provedores de servi\c{c}o de software largamente distribu\'{i}dos~\cite{Perin-Souza:2010:AMA:1909623.1909668}.

Observamos que caso tiv\'{e}ssemos definido termos de QoS em particular, como dependabilidade ou desempenho no termo de busca, outros grupos poder\'{i}am ter-se destacado no \^{a}mbito brasileiro, como \'{e} o caso do grupo de Rubira et al. (Unicamp). Mas por quest\~{a}o de viabilidade do estudo, restringimo-nos ao termo de qualidade de servi\c{c}o.
\subsection{Questão de Pesquisa 4}

\emph{Qual o foco da contribuição de pesquisa realizada em SOC e relacionada com qualidade de serviço? }

Com o intuito de responder a esta questão, foi feita uma avalia\c c\~{a}o da distribui\c c\~{a}o dos 
artigos em rela\c c\~{a}o ao tipo de pesquisa (conforme discutido na Se\c c\~{a}o~\ref{sec:review_method}). 
Os \emph{bubble plots} nas Figuras~\ref{} e~\ref{}  apresentam tal distibui\c c\~{a}o, novamente sendo importante 
ressaltar que o n\'{u}mero total de artigos nos gr\'{a}ficos \'{e} superior ao n\'{u}mero total de artigos analisados--- 
uma vez que alguns artigos apresentam contribui\c c\~{o}es tanto em termos de uma nova solu\c c\~{a}o proposta 
quanto em termos de avalia\c c\~{a}o e/ou valida\c c\~{a}o. Por exemplo, Huang et al. prop\~{o}e um 
modelo estoc\'{a}stico para representar e raciocinar sobre dependabilidade em um ambiente de SOC, ao mesmo tempo 
que valida formalmente tal proposta por meio de provas de teoremas~\cite{huang:scc2011}.

Esta investiga\c c\~{a}o revelou que 21 artigos (como por exemplo~\cite{jeong:fqs2009,ardagna:jss2010}) 
contribuem com uma nova solu\c c\~{a}o para 
lidar com qualidade de servi\c co em SOC ao mesmo tempo que apresentam uma investiga\c c\~{a}o tanto 
em termos de avalia\c c\~{a}o quanto em termos de valida\c c\~{a}o. Al\'{e}m disso, 80 
artigos s\~{a}o propostas de solu\c c\~{o}es que apresentam 
avalia\c c\~{o}es em termos de estudos de casos mais simples, n\~{a}o compreendendo uma 
valida\c c\~{a}o da(s) t\'{e}cnica(s) proposta(s). Entre esses artigos podemos 
citar~\cite{filieri:faa2012, pernici:services2011,nascimento:splc2011}. Finalmente, 27 artigos apresentam, 
al\'{e}m de uma nova solu\c c\~{a}o relacionada \`{a} qualidade de servi\c cos em SOC, 
uma s\'{o}lida valida\c c\~{a}o (e.g.~\cite{huang:scc2011,binshtok:icsoc2009}).    
Por outro lado, 102 artigos (aproximadamente 40\% do total) prop\~{o}em novas 
solu\c c\~{o}es sem apresentar avalia\c c\~{a}o ou valida\c c\~{a}o consistentes (\cite{balfagih:icime2011,fiadeiro:fac2011,khazankin:scc2011}). 
Enquanto que apenas 4 artigos focam na avalia\c c\~{a}o e/ou valida\c c\~{a}o de 
propostas existentes~\cite{voelz:edoc2010,moayed:icsea2010,cavallo:pesos2010,banerjee:isorcw2011}.  

Esses n\'{u}meros revelam que a \'{a}rea de pesquisa de qualidade de servi\c co em 
SOC ainda est\'{a} em uma fase 
de amadurecimento, onde um percentual significativo das contribui\c c\~{o}es 
simplesmente apresentam novas abordagens ou fazem compara\c c\~{o}es envolvendo a 
pr\'{o}pria t\'{e}cnica proposta--- o que pode levar a conclus\~{o}es tendenciosas. A quantidade 
de artigos que visam avaliar ou validar t\'{e}cnicas existentes, algo recomendado antes 
de se iniciar a concep\c c\~{a}o de uma nova solu\c c\~{a}o, \'{e} praticamente insignificante. 
 


\section{Discuss\~{a}o Sobre os Resultados}

Com base nos resultados evidenciados por esse MS, \'{e} poss\'{i}vel
confirmar v\'{a}rios aspectos relevantes. O resultado desse mapeamento
mostra que a maior parte dos trabalhos publicados lida com
monitoramento e adapta\c{c}\~{a}o (\MonitoramentoAdaptacao), seguida de modelos de QoS
\& linguagens (\ModelosdeQoSeLinguagens), descoberta \& sele\c{c}\~{a}o (\DescobrimentoeSelecao),
composi\c{c}\~{a}o (\Composicao). Esses resultados indicam o foco dado a aspectos n\~{a}o funcionais, din\^{a}micos e que podem sofrer varia\c{c}\~{o}es devido a concorrência e possíveis falhas dos servi\c{c}os em tempo de execu\c{c}\~{a}o. Mas tamb\'{e}m indicam o foco em propostas que visam a representa\c{c}\~{a}o e a escolha da
melhor configura\c{c}\~{a}o de modo a atender aos níveis globais
desej\'{a}veis ou necess\'{a}rios de QoS para um conjunto de
servi\c{c}os. 
No contexto de SOC, os dados gerais nos induzem a
concluir que desempenho, disponibilidade e confiabilidade s\~{a}o
prioridade como atributos de QoS em SOC. Seguran\c{c}a,
escalabilidade e custo, no entanto, tiveram menor evid\^{e}ncia no trabalhos analisados em nosso estudo. 
Um futuro estudo com base em cada um desses atributos separadamente pode servir de base para se ter um panorama
mais espec\'{i}fico em cada uma dessas \'{a}reas. Mas entendemos que, devido \`{a} abrang\^{e}ncia de cada um desses t\'{o}picos,
tornar-se-ia invi\'{a}vel tratar dessa quest\~{a}o em nosso estudo. 

Quanto aos grupos que mais contribu\'{i}ram para o contexto de
mapeamento est\~{a}o 3 principais grupos: (1) o S-Cube (com foco em
monitoramento \& adapta\c{c}\~{a}o, assim como composi\c{c}\~{a}o
din\^{a}mica), (2) Menasc\'{e} et al. (com foco em descobrimento \&
sele\c{c}\~{a}o, assim como monitoramento \& adapta\c{c}\~{a}o) e (3)
Lin et al (composi\c{c}\~{a}o din\^{a}mica e
adapta\c{c}\~{a}o). Percebemos tamb\'{e}m que suas
contribui\c{c}\~{o}es se encaixam na \'{a}reas de SOC que mais de
destacaram em nosso estudo. 
Finalmente, em rela\c c\~{a}o ao tipo de pesquisa em
qualidade de servi\c{c}o em SOC, consideramos que a mesma ainda
est\'{a} em uma fase de amadurecimento, 
onde um percentual significativo das contribui\c{c}\~{o}es
simplesmente apresentam 
novas abordagens ou fazem compara\c{c}\~{o}es envolvendo a 
própria t\'{e}cnica proposta. Na pr\'{o}xima se\c c\~{a}o apresentamos
algumas amea\c cas \`{a} validade do nosso estudo, bem como as
estra\'{e}gias que seguimos para que as mesmas
fossem contornadas.

Por fim, conforme observamos na Seção \ref{sub:QP4}, pesquisas em qualidade de servi\c co em SOC ainda est\~{a}o em uma fase de amadurecimento naquilo que se refere ao uso e adoção em casos de uso reais das propostas. Dado que 44.8\% dos trabalhos mapeados n\~{a}o realizam validação e que apenas 9\% conduzem uma avaliação real, não podemos confirmar nossa hipótese inicial de que pesquisas 
relativas a QoS em SOC estejam próximas a um patamar de amadurecimento, conforme o ciclo de maturação de Redwine et al.~\cite{redwine:icse1985}.