\subsection{Quest\~{a}o de Pesquisa 3}
\emph{Quais atributos de qualidade são frequentemente considerados nos estudos abordados?}

Para responder a essa pergunta, olhamos a distribui\c{c}\~{a}o dos artigos no eixo de QoS, a faceta de contexto. A Figura~\ref{Fig:bubbleplot} apresenta o diagrama ilustrando essa distribui\c{c}\~{a}o. Vale ressaltar que, como cada artigo pode tratar de m\'{u}ltiplos atributos de QoS, a soma total do n\'{u}mero de artigos mapeados em cada um dos atributos n\~{a}o totalizar\'{a} o n\'{u}mero total de artigos inclu\'{i}dos no mapeamento. Por exemplo, o artigo~\cite{DBLP:journals/tse/CalinescuGKMT11} lida com os atributos de disponbilidade, desempenho e confiabilidade. 

O mapa mostra que SLA (90\% dos artigos) \'{e} o que predomina, seguido de desempenho (59.2\% dos artigos), disponibilidade (49.6\% dos artigos) e confiabilidade (38\% dos artigos). Os atributos menos observados s\~{a}o custo (19.6\% dos artigos), seguran\c{c}a (16.4\% dos artigos) e escalabilidade (13.2\% dos artigos). Os atributos que n\~{a}o se enquadraram especificamente em nenhum desses foram classificados em outros (9.2\% dos artigos) como aqueles que envolvem outros atributos de qualidade como, por exemplo, \cite{6036406} que define um crit\'{e}rio de sele\c{c}\~{a}o de servi\c{c}o conforme sua reputa\c{c}\~{a}o. 

A partir desses resultados, pode-se notar que, no contexto de SOC, os termos mais relacionados a QoS s\~{a}o SLA, desempenho, disponibilidade e confiabilidade, com bastante \^{e}nfase em SLA. No entanto, observamos muitas vezes que os trabalhos mencionavam QoS sem explicitar qual atributo em particular estava em quest\~{a}o. Nesses casos, com base nas m\'{e}tricas utilizadas, classificamos como SLA por ser a op\c{c}\~{a}o mais pr\'{o}xima naquele contexto. Dessa forma, considerando essa classifica\c{c}\~{a}o do SLA como poss\'{i}vel lacuna de clareza nos trabalhos avaliados, os dados gerais nos induzem a concluir que desempenho, disponibilidade e confiabilidade s\~{a}o prioridade como atributos de QoS em SOC. No entanto, o mesmo n\~{a}o pode ser conclu\'{i}do para seguran\c{c}a, escalabilidade e custo. 

%Com essa quest\~{a}o de pesquisa, pretendemos tamb\'{e}m obter quais s\~{a}o os autores com maior n\'{u}mero de publica\c{c}\~{o}es relacionadas ao contexto desse estudo.