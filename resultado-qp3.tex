\subsection{Questão de Pesquisa 3}\label{sec:QP3}

\textbf{Quais s\~{a}o os estudos existentes que mais tem impulsionado QoS em SOC?}
\\[0.01in]

%Com essa quest\~{a}o de pesquisa, pretendemos tamb\'{e}m obter quais s\~{a}o os autores com maior n\'{u}mero de publica\c{c}\~{o}es relacionadas ao contexto desse estudo.

Para responder a esta quest\~{a}o, inicialmente identificamos alguns dos principais grupos de pesquisa que mais contribu\'{i}ram com pesquisas em SOC no contexto de QoS. Esses grupos foram classificados conforme segue: 

\textbf{Grupo S-Cube} -- Identificamos que dos quatros grupos que mais contribu\'{i}ram no \^{a}mbito desse estudo est\~{a}o pesquisadores cuja afilia\c{c}\~{a}o est\'{a} inserida direta ou indiretamente no contexto do grupo europeu S-Cube~\cite{SCube-FINALREPORT}. No per\'{i}odo do nosso estudo, o grupo S-Cube contribuiu com 10 publica\c{c}\~{o}es relevantes para esse contexto. S\~{a}o os seguintes autores Schahram Dustdar (Vienna University of Techonology) e Raffaela Mirandola (Politecnico di Milano) com seus colaboradores Valeria Cardelini e Emiliano Casalicchio ambos de Universit\'{a} di Roma ``Tor Vergata''. Em particular, percebemos que Mirandola e seus colaboradores t\^{e}m contribu\'{i}do em pesquisas relacionadas a monitoramento e adapta\c{c}\~{a}o no contexto de confiabilidade, disponibilidade e desempenho como pode ser percebido com as publica\c{c}\~{o}es \cite{Cardellini:2009:QRA:1595696.1595718, Calinescu:2011:DQM:1990772.1991012, Ardagna:2010:POS:1814581.1814611, 10.1109/TSE.2011.68, Cardellini:2009:TSD:1692867.1692870}. Dustdar e seus colaboradores, por sua vez, t\^{e}m contribu\'{i}do em composi\c{c}\~{a}o de servi\c{c}os em ambientes din\^{a}micos no escopo de SLA, com destaque para o  VRESCO (\emph{Vienna Runtime Environment for Service-Oriented Computing})~\cite{5467022}. 
% Em particular, segundo a base da IEEE Xplore mostra que essa contribui\c{c}\~{a}o recebeu, 
% at\'{e} o momento desse estudo, em torno de 42 cita\c{c}\~{o}es desde Setembro de 2010.

\textbf{Daniel Menasce et al.} -- Menasc\'{e} e seus colaboradores t\^{e}m tradicionalmente contribu\'{i}do com pesquisas relativas a QoS, em particular no \^{a}mbito de desempenho, incluindo os diversos f\'{o}runs da \'{a}rea~\cite{DBLP:journals/tse/MenasceG00, Menasce:2001:CPW:560806}. No per\'{i}odo do nosso estudo, Menasc\'{e} et al. contribu\'{i}ram com 6 publica\c{c}\~{o}es e destacaram-se no contexto de SLA nas \'{a}reas de descobrimento \& sele\c{c}\~{a}o e monitoramento \& adapta\c{c}\~{a}o~\cite{5696721, DBLP:MenasceCD10, 5552741}. Em~\cite{5696721}, Menasc\'{e} et al. contribuem com o SASSY, um arcabou\c{c}o que gera automaticamente arquiteturas de software candidatas e seleciona aquela que melhor se adequa ao objetivo de QoS. Em~\cite{DBLP:MenasceCD10}, eles apresentam um algoritmo que encontra a solu\c{c}\~{a}o para otimiza\c{c}\~{a}o na busca de provedores de servi\c{c}o com restri\c{c}\~{o}es de custo e tempo de execu\c{c}\~{a}o. 
%Vale ressaltar que esse trabalho teve 39 cita\c{c}\~{o}es at\'{e} o momento desse estudo e foi 
%realizado em coopera\c{c}\~{a}o com Casalicchio, que tamb\'{e}m teve colabora\c{c}\~{o}es com o grupo do S-Cube.  

\textbf{Kwei-Jay Lin et al.} -- O grupo de Lin (University of California, Irvine) tem contribu\'{i}do essencialmente no contexto de SLA no \^{a}mbito de composi\c{c}\~{a}o (din\^{a}mica) e adapta\c{c}\~{a}o de servi\c{c}os e contribu\'{i}ram com 4 publica\c{c}\~{o}es no per\'{i}odo do mapeamento. Particular destaque para  as contribui\c{c}\~{o}es \cite{Lin:2009:EAS:1602932.1603224, Lin:2010:DIS:1861294.1861332, Zhai:2009:SMS:1586636.1586972} que lidam com reconfigura\c{c}\~{a}o de servi\c{c}os em Service-Oriented Architecture, com restri\c{c}\~{o}es de QoS fim-a-fim. 

%No Brasil em particular, o grupo que mais tem contrbu\'{i}do com o tema no per\'{i}odo do estudo:

%\textbf{Ricardo Rabelo et al.} --  Rabelo e seu grupo tem contribu\'{i}do com mecanismos de descoberta de servi\c{c}o no contexto de QoS tanto para alavancar servic\c{c}os de software compartilhados sobre o que chamam de redes colaborativas~\cite{conf/ifip5-5/Perin-SouzaR11} quanto para integrar BPM e SOA no contexto de provedores de servi\c{c}o de software largamente distribu\'{i}dos~\cite{Perin-Souza:2010:AMA:1909623.1909668}.

%Observamos que caso tiv\'{e}ssemos definido termos de QoS em particular, como dependabilidade ou desempenho no termo de busca, outros grupos poder\'{i}am ter-se destacado no \^{a}mbito brasileiro, como \'{e} o caso do grupo de Rubira et al. (Unicamp). Mas por quest\~{a}o de viabilidade do estudo, restringimo-nos ao termo de qualidade de servi\c{c}o.