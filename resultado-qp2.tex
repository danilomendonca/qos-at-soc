\subsection{Quest\~{a}o de Pesquisa 2}
\textbf{Quais atributos de qualidade são frequentemente considerados nos estudos abordados?}
\\[0.01in]

Para responder a essa pergunta, olhamos a distribui\c{c}\~{a}o dos artigos no eixo de QoS, a faceta de contexto. O eixo vertical da Figura~\ref{fig:bubbleplot-QoSSOC} apresenta os resultados dessa distribui\c{c}\~{a}o. Vale ressaltar que, como cada artigo pode tratar de m\'{u}ltiplos atributos de QoS, a soma total do n\'{u}mero de artigos mapeados em cada um dos atributos n\~{a}o totalizar\'{a} o n\'{u}mero de artigos inclu\'{i}dos no mapeamento. Por exemplo, o artigo~\cite{DBLP:journals/tse/CalinescuGKMT11} lida com os atributos de disponbilidade, desempenho e confiabilidade. 

O mapa mostra que SLA (\SLA) \'{e} o que predomina, seguido de desempenho (\Desempenho), disponibilidade (\Disponibilidade) e confiabilidade (\Confiabilidade). Os atributos menos observados s\~{a}o custo (\Custo), seguran\c{c}a (\Seguranca) e escalabilidade (\Escalabilidade). Os atributos que n\~{a}o se enquadraram especificamente em nenhum desses foram classificados em outros (\Outros) como aqueles que envolvem outros atributos de qualidade como, por exemplo, \cite{6036406} que define um crit\'{e}rio de sele\c{c}\~{a}o de servi\c{c}o conforme sua reputa\c{c}\~{a}o. 

A partir desses resultados, pode-se notar que, no contexto de SOC, os termos mais relacionados a QoS s\~{a}o SLA, desempenho, disponibilidade e confiabilidade, com bastante \^{e}nfase em SLA. Este resultado se justifica uma vez que diversas contribuições para monitoramento e adaptação, descoberta, composição e seleção de serviços, entre outras, não especificam atributos de qualidade, deixando em aberto quais métricas serão usadas ao se instanciar a proposta. Com relação aos dados específicos, conclui-se que desempenho, disponibilidade e confiabilidade s\~{a}o prioridade como atributos de QoS em SOC. Seguran\c{c}a, escalabilidade e custo, no entanto, tiveram menor evid\^{e}ncia no trabalhos analisados em nosso estudo, sendo consideradas possíveis lacunas de pesquisa.