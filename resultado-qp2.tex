\subsection{Quest\~{a}o de Pesquisa 2}
\emph{Quais as áreas de SOC são mais frequentemente pesquisadas no contexto de qualidade de serviços?}

%A QP2 tem como objetivo trazer uma perspectiva do cen\'{a}rio das pesquisas em Computa\c{c}\~{a}o Orientada a Servi\c{c}o com foco em QoS atualmente. Para responder a essa pergunta, precisamos primeiramente definir quais s\~{a}o as \'{a}reas que melhor caracterizam as diversas contribui\c{c}\~{o}es de pesquisa em SOC. Da\'{i} ent\~{a}o realizamos a classifica\c{c}\~{a}o.

Para responder a essa pergunta primeiramente identificamos as principais caracter\'{i}sticas de SOC. Encontramos no projeto europeu S-Cube a fundamenta\c{c}\~{a}o mais clara para definir essas caracter\'{i}sticas~\cite{SCube-FINALREPORT} e, portanto, adotamos como refer\^{e}ncia para definir a faceta da contribui\c{c}\~{a}o. Para atender ao foco desse estudo, no entanto, adaptamos a estrutura definidas no projeto S-Cube e obtemos as seguintes caracter\'{i}sticas: Ciclo de Vida, Composi\c{c}\~{a}o, Coordena\c{c}\~{a}o \& Comunica\c{c}\~{a}o, Descoberta \& Sele\c{c}\~{a}o, Modelos de QoS, Monitoramento \& Adapta\c{c}\~{a}o. 

A Figura~\ref{Fig:bubbleplot} apresenta o diagrama ilustrando distribui\c{c}\~{a}o dos artigos no eixo de SOC, a faceta da contribui\c{c}\~{a}o. Assim como os resultados observados nas outras facetas, vale ressaltar que, como cada artigo pode tratar de m\'{u}ltiplas caracter\'{i}siticas de SOC, a soma total do n\'{u}mero de artigos mapeados em cada um dos atributos n\~{a}o totalizar\'{a} o n\'{u}mero total de artigos inclu\'{i}dos no mapeamento. Por exemplo, o artigo~\cite{A QoS-aware fault tolerant middleware for dependable service composition} lida com os composi\c{c}\~{a}o, modelos de QoS \& linguagens assim como monitoramento \& adapta\c{c}\~{a}o. 

O resultado desse mapeamento mostra que a maior parte dos trabalhos publicados, lida com monitoramento e adapta\c{c}\~{a}o (34,8\%),seguida de modelos de QoS (33,2\%), descoberta \& sele\c{c}\~{a}o (28\%),  composi\c{c}\~{a}o (27,7\%) ciclo de vida (14,4\%)...

