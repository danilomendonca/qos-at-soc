\subsection{Quest\~{a}o de Pesquisa 2}
\emph{Quais as áreas de SOC são mais frequentemente pesquisadas no contexto de qualidade de serviços?}

%A QP2 tem como objetivo trazer uma perspectiva do cen\'{a}rio das pesquisas em Computa\c{c}\~{a}o Orientada a Servi\c{c}o com foco em QoS atualmente. Para responder a essa pergunta, precisamos primeiramente definir quais s\~{a}o as \'{a}reas que melhor caracterizam as diversas contribui\c{c}\~{o}es de pesquisa em SOC. Da\'{i} ent\~{a}o realizamos a classifica\c{c}\~{a}o.

Para responder a essa pergunta primeiramente identificamos as principais caracter\'{i}sticas de SOC. Encontramos no projeto europeu S-Cube a fundamenta\c{c}\~{a}o mais clara para definir tais caracter\'{i}sticas~\cite{SCube-FINALREPORT} e, portanto, o adotamos como refer\^{e}ncia para definição da faceta da contribui\c{c}\~{a}o. Visando atender ao foco desse estudo, no entanto, adaptamos a estrutura definida no projeto S-Cube. Tal classificação buscou classificar partes relevantes de SOC que agrupassem trabalhos de pesquisa cujas áreas de contribuições sejam correlatas. A estrutura proposta foi validada a medida em as publicações puderam ser mapeadas em no mínimo um desses atributos, não havendo exceções. 

%e obtivemos as seguintes caracter\'{i}sticas: Ciclo de Vida, Composi\c{c}\~{a}o, Coordena\c{c}\~{a}o \& Comunica\c{c}\~{a}o, Descoberta \& Sele\c{c}\~{a}o, Modelos de QoS, Monitoramento \& Adapta\c{c}\~{a}o. 

A Figura~\ref{Fig:bubbleplot} apresenta o diagrama ilustrando a distribui\c{c}\~{a}o dos artigos no eixo de SOC, a faceta da contribui\c{c}\~{a}o. Assim como os resultados observados nas outras facetas, vale ressaltar que os atributos mapeados n\~{a}o s\~{a}o mutualmente excludentes, mas sim complementares, visto que representam partes de SOC. Por exemplo, o artigo~\cite{A QoS-aware fault tolerant middleware for dependable service composition} lida com os composi\c{c}\~{a}o, modelos de QoS \& linguagens assim como monitoramento \& adapta\c{c}\~{a}o. Existe dentre eles alguma sobreposi\c{c}\~{a}o, notavelmente entre composição \& coordenação. No entanto, notou-se que o termo coordenação tem um significado mais amplo que a própria coordenação de serviços e também envolve aspectos da comunicação entre provedores e consumidores de serviços em geral, independente de composição de serviços. Dessa forma adotamos, em conjunção a coordenação, o termo comunicação.

O resultado desse mapeamento mostra que a maior parte dos trabalhos publicados lida com monitoramento e adapta\c{c}\~{a}o (\MonitoramentoAdaptacao), seguida de modelos de QoS (\ModelosdeQoSeLinguagens), descoberta \& sele\c{c}\~{a}o (\DescobrimentoSelecao),  composi\c{c}\~{a}o (\Composicao), ciclo de vida (\Ciclodevida) e finalmente coordenação com \CoodenacaoComunicacao.

Tais resultados indicam a relevância do monitoramento e adaptação em sistemas baseados em serviços, sobretudo quando aspectos não funcionais ou de qualidade são dinâmicos e podem sofrer variações não só na camada de transporte, mas também devido ao acesso concorrente, visto que diferentes clientes podem compartilhar dos mesmos serviços, sobretudo quando a provisão desses é delegada a terceiros. Este modelo de terceirização de serviços é amplamente utilizado pela Computação em Nuvens \cite{}.
