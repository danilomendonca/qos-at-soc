%TCIDATA{LaTeXparent=0 0 sbrc_2013.tex}
%Introdução

\section{Introdu\c{c}\~{a}o}\label{sec:introduction}

A \emph{Computa\c c\~{a}o Orientada a Servi\c cos} (ou \emph{Service Oriented Computing} -- SOC) emergiu como 
um novo paradigma de computa\c c\~{a}o que utiliza servi\c cos como componentes b\'{a}sicos para o desenvolvimento 
de aplica\c c\~{o}es~\cite{papazoglou:cacm2003}, e tendo como blocos b\'{a}sicos as opera\c c\~{o}es de infraestrutura providas 
pela arquiteturas orientadas a servi\c co (como publica\c c\~{a}o, sele\c c\~{a}o, descoberta e composi\c c\~{a}o).  
Um dos objetivos em SOC \'{e} a automa\c c\~{a}o e integra\c c\~{a}o de processos de neg\'{o}cio e/ou cient\'{i}ficos que podem envolver 
diferentes organiza\c c\~{o}es. Dada as necessidades de orquestra\c c\~{a}o e coreografia e \`{a}s  caracter\'{i}ticas tipicamente distribu\'{i}das e heterog\^{e}neas encontradas 
nos cen\'{a}rios aos quais SOC foi proposta, novas preocupa\c c\~{o}es relacionadas \`{a} garantia de qualidade de 
servi\c co (\emph{Quality of Service} --- QoS) motivaram um interesse crescente dos 
pesquisadores na \'{a}rea. 

Quando novas \'{a}reas de pesquisa emergem, particularmente quando amparadas por 
certo apelo da ind\'{u}stria, \'{e} natural que uma quantidade significativa de 
contribui\c c\~{o}es cient\'{i}ficas foquem primariamente em novas propostas de solu\c c\~{a}o 
(fases que compreendem o momento da formula\c c\~{a}o de um paradigma por uma comunidade ainda restrita 
ao momento do seu refinamento e explora\c c\~{a}o por uma audi\^{e}ncia mais ampla~\cite{redwine:icse1985}).   
Por outro lado, com o amadurecimento da pesquisa realizada em uma \'{a}rea, espera-se o aprofundamento 
em termos de evid\^{e}ncias relacionadas \`{a} aplicabilidade das t\'{e}cnicas propostas, para que em seguida as mesmas 
sejam adotadas pela comunidade--- Redwine et al. argumenta que 
esse ciclo de \emph{matura\c c\~{a}o} dura aproximadamente 20 anos para a \'{a}rea de tecnologia, em 
particular para a \'{a}rea de software~\cite{redwine:icse1985}.

Apesar da pesquisa em SOC ter emergido h\'{a} pouco mais de 10 anos 
(uma edi\c c\~{a}o especial da \emph{Communications of ACM}~\cite{papazoglou:cacm2003} sobre SOC foi publicada em 2003), 
n\~{a}o existe um mapeamento que possibilite identificar o est\'{a}gio da pesquisa realizada na \'{a}rea de qualidade de 
servi\c cos em SOC, dificultando a identifica\c c\~{a}o de tend\^{e}ncias e oportundiades de pesquisa e tornando mais 
lenta a ado\c c\~{a}o das t\'{e}cnicas propostas. Nossa hip\'{o}tese inicial \'{e} que o foco de pesquisa em QoS / SOC deve 
estar alcan\c cando o patamar de amadurecimento em que uma quantidade significativa de trabalhos objetiva evidenciar os 
benef\'{i}cios das solu\c c\~{o}es propostas por meio de estudos experimentais e valida\c c\~{o}es. Com o intuito de investigar tal hip\'{o}tese, 
este artigo apresenta os resultados de um mapeamento sistem\'{a}tica de estudos (cujo protocolo \'{e} descrito na Se\c c\~{a}o~\ref{sec:review_method}) 
sobre a \'{a}rea, revelando, al\'{e}m de da maturidade da pesquisa, quais s\~{a}os os principais atributos de qualidade 
investigados e grupos de pesquisa que atuam na \'{a}rea. 

Organizamos nossa investiga\c{c}\~{a}o em termos de quatro quest\~{o}es de pesquisa (apresentadas inicialmente na 
Se\c c\~{a}o~\ref{sec:questoesPesquisa}). Tais quest\~{o}es s\~{a}o respondidas ao longo de toda a Se\c c\~{a}o~\ref{sec:resultados}. 
Finalmente, consolidamos nossas observa\c c\~{o}es e apresentamos as amea\c cas para a validade do estudo na 
Se\c c\~{a}o~\ref{sec:conclusao}. 
 
 


% SOC, SOA, \textit{Web Services}, SOAP, REST, Orientação a Serviços, Computação em Nuvem. Nos últimos anos esses e outros acrônimos tornaram-se frequentes na tecnologia da informação. O surgimento de um novo paradigma, impulsionado pelo amadurecimento da internet e pela proximidade crescente entre negócios e TI, criou novos caminhos e oportunidades para trabalhos de desenvolvimento e pesquisa. Nesse sentido, um grande número de estudos foram e vem sendo conduzidos com foco nos diversos aspectos da computação orientada a serviços, tais quais arquitetura, modelos, métodos, processos, ferramentas diversas, frameworks, métricas, problemas solucionados e ainda vigentes. Desta forma, a intenção daqueles interessados em iniciar suas atividades na área fica comprometida pela dificuldade em se obter informações claras sobre o atual estado da arte, os desafios e os temas mais abordados e aqueles com deficit de pesquisas. Esses dados são cruciais para que esforços sejam bem direcionados e para que a ciência caminhe em cooperação e com eficiência.

% Um mapeamento sistemático de estudos visa classificar de forma sistemática e ampla um conjunto de estudos. Dada a grande quantidade de publicações no escopo da orientação a serviços, sua metodologia ágil e que permite a análise de um maior número de estudos justifica sua escolha em detrimento de outras metodologias, como o \textit{Systematic Literature Review} \cite{Petersen_Feldt_Mujtaba_Mattsson_2007}. Essa última exige uma análise minuciosa e detalhada de cada publicação, o que requer um esforço considerável e inviabiliza a inclusão de um grande número de publicações num quadro de poucos pesquisadores. Assim, dados os fatos citados e o interesse em se obter uma classificação ampla e significativa da ciência relacionada à orientação a serviços, de caráter inicial e que irá servir de subsídio a outros estudos, este trabalho de conclusão de curso em Engenharia de Redes de Comunicação realiza um mapeamento sistemático de estudos abrangendo a orientação a serviços. 

% Segundo \cite{Papazoglou:2007:SOA:1265289.1265298}, devido ao crescente acordo na implementação e gerência de aspectos funcionais de serviços, tal qual a adoção de WSDL para a descrição, SOAP para troca de mensagens, ou WS-BPEL para a composição, os interesses de pesquisadores estão se voltando aos aspectos não funcionais de aplicações orientadas a serviços. Visando essa constatação, nosso mapeamento irá concentrar-se na questão de qualidade, ou aspectos não funcionais, sobretudo a qualidade de serviços, termo aqui empregado de forma literal e posterior ao termo QoS, uma vez que os principais agentes do paradigma em questão são, coincidentemente, denominados serviços. Ademais, o ambiente proposto pelo SOC está sujeito a condições particulares diferentes daquelas já estudadas e conhecidas em outros paradigmas, havendo variáveis que elevam a complexidade da análise de parâmetros de qualidade, tanto na fase de planejamento quanto em fase de execução por meio do monitoramento e da gerência dos serviços, sendo esse um obstáculo sólido à adoção de arquiteturas como o SOA. Nesse sentido, o presente estudo visa mapear as publicações relacionadas a essas questões, contemplando cenários com ou sem o uso de SOA, proporcionando uma redução da incerteza quanto ao atual estado de desenvolvimento da ciência contribuinte ao tema abordado e quanto aos desafios e avanços já conquistados.

% No que tange a trabalhos relacionados, este trabalho possui características inéditas dentro do campo de QoS em SOC. Dentre as referências atuais e mais relevantes no que concerne modelos de QoS em SOC pode ser encontrado em \cite{pernici}, produzido pelo projeto europeu S-CUBE~\cite{scube}. Esta, além desse citado, produziu uma coletânea de outros relatórios e trabalhos que analisam publicações em praticamente todas as eferas do SOC. Entretanto, trata-se de trabalhos de Systematic Literature Review, visto que analisam profundamente as publicações envolvidas na área e as restringe àquelas com maior qualidade e aceitação, indicando as vantagens e limitações das propostas analisadas. Em contraste, o MS proposto abrange um número maior de estudos, trazendo informações em categorias mais amplas e que possibilitam a melhor análise geral da pesquisa relacionada. São dados amplos mas sensíveis para a compreensão do estado da ciência envolvida com os aspectos qualitativos do SOC.

% Em geral, o resultado de um estudo de mapeamento é um mapa visual classificando os resultados obtidos. Em particular, acreditamos que esse mapeamento pode beneficiar o estado da arte e da pr\'{a}tica em Computa\c{c}\~{a}o Orientada a Servi\c{c}os identificando tend\^{e}ncias e oportunidades para transfer\^{e}ncia de conhecimento. Considerando tamb\'{e}m o tamanho e abrang\^{e}ncia dessa \'{a}rea outros  objetivos do presente mapeamento de estudos s\~{a}o tamb\'{e}m esclarecer o paradigma da orientação a serviços no contexto de qualidade de servi\c{c}os (QoS) por meio de uma classificação ampla e sistemática, obtendo informações sobre frequências de publica\c{c}\~{o}es, \'{a}reas e t\'{o}picos de pesquisa, enfoques, tipos de contribui\c{c}\~{o}es de pesquisa dadas, os agentes e f\'{o}runs envolvidos.

% As demais se\c{c}\~{o}es desse artigo est\~{a}o organizadas da seguinte forma: Se\c{c}\~{a}o \ref{?}.Se\c{c}\~{a}o \ref{?}. Se\c{c}\~{a}o \ref{?}. Finalmente, na Se\c{c}\~{a}o \ref{?} apresentamos um resumo das nossas descobertas e provemos tamb\'{e}m algumas discuss\~{o}es finais. 

% %Neste mapeamento, deve-se prezar pelo uso de ferramentas de apoio e que agilizem os procedimentos sistemáticos a serem seguidos. Não é seu objetivo avaliar qualitativamente os trabalhos de pesquisa classificados, mantendo a análise a um nível menos detalhado e que permitirá a inclusão dessas publicações em categorias abrangentes e significativas e que não exijam a minuciosa análise de cada uma delas. 

% %Em relação aos atributos de QoS, tem-se por objetivo compreender que tipo de intervenções vem sendo feitas para a absorção de aspectos de qualidade em SOC e sua melhoria, identificando quais tipos de pesquisa,  de contribuição e quais atributos ou contextos tem maior importância, quais representam os maiores desafios para a concretização da adoção deste paradigma e quais são pouco abordados. Entre os variados e numerosos tipos de atributos que atendem a diversos modelos, este trabalho visa aqueles cuja definição encontra-se bem difundida e aceita entre os trabalhos relacionados. São eles o desempenho, a disponibilidade, a confiabilidade, a segurança, a modificabilidade, a testabilidade, a escalabilidade, o custo e outros para que os demais atributos sejam representados agrupadamente. Entre os tipos de pesquisa aceitos, estão os de avaliação, solução, validação, filosófico e de experiência pessoal. Por fim, entre os tipos de contribuições dadas, estão as de modelo, método, processo, ferramenta e métricas.
