%TCIDATA{LaTeXparent=0 0 sbrc_2013.tex}
%Introdução

\section{Introdu\c{c}\~{a}o}\label{sec:introduction}

SOC, SOA, \textit{Web Services}, SOAP, REST, Orientação a Serviços, Computação em Nuvem. Nos últimos anos esses e outros acrônimos tornaram-se frequentes na tecnologia da informação. O surgimento de um novo paradigma, impulsionado pelo amadurecimento da internet e pela proximidade crescente entre negócios e TI, criou novos caminhos e oportunidades para trabalhos de desenvolvimento e pesquisa. Nesse sentido, um grande número de estudos foram e vem sendo conduzidos com foco nos diversos aspectos da computação orientada a serviços, tais quais arquitetura, modelos, métodos, processos, ferramentas diversas, frameworks, métricas, problemas solucionados e ainda vigentes. Desta forma, a intenção daqueles interessados em iniciar suas atividades na área fica comprometida pela dificuldade em se obter informações claras sobre o atual estado da arte, os desafios e os temas mais abordados e aqueles com deficit de pesquisas. Esses dados são cruciais para que esforços sejam bem direcionados e para que a ciência caminhe em cooperação e com eficiência.

Um mapeamento sistemático de estudos visa classificar de forma sistemática e ampla um conjunto de estudos. Dada a grande quantidade de publicações no escopo da orientação a serviços, sua metodologia ágil e que permite a análise de um maior número de estudos justifica sua escolha em detrimento de outras metodologias, como o \textit{Systematic Literature Review} \cite{Petersen_Feldt_Mujtaba_Mattsson_2007}. Essa última exige uma análise minuciosa e detalhada de cada publicação, o que requer um esforço considerável e inviabiliza a inclusão de um grande número de publicações num quadro de poucos pesquisadores. Assim, dados os fatos citados e o interesse em se obter uma classificação ampla e significativa da ciência relacionada à orientação a serviços, de caráter inicial e que irá servir de subsídio a outros estudos, este trabalho de conclusão de curso em Engenharia de Redes de Comunicação realiza um mapeamento sistemático de estudos abrangendo a orientação a serviços. 

Segundo \cite{Papazoglou:2007:SOA:1265289.1265298}, devido ao crescente acordo na implementação e gerência de aspectos funcionais de serviços, tal qual a adoção de WSDL para a descrição, SOAP para troca de mensagens, ou WS-BPEL para a composição, os interesses de pesquisadores estão se voltando aos aspectos não funcionais de aplicações orientadas a serviços. Visando essa constatação, nosso mapeamento irá concentrar-se na questão de qualidade, ou aspectos não funcionais, sobretudo a qualidade de serviços, termo aqui empregado de forma literal e posterior ao termo QoS, uma vez que os principais agentes do paradigma em questão são, coincidentemente, denominados serviços. Ademais, o ambiente proposto pelo SOC está sujeito a condições particulares diferentes daquelas já estudadas e conhecidas em outros paradigmas, havendo variáveis que elevam a complexidade da análise de parâmetros de qualidade, tanto na fase de planejamento quanto em fase de execução por meio do monitoramento e da gerência dos serviços, sendo esse um obstáculo sólido à adoção de arquiteturas como o SOA. Nesse sentido, o presente estudo visa mapear as publicações relacionadas a essas questões, contemplando cenários com ou sem o uso de SOA, proporcionando uma redução da incerteza quanto ao atual estado de desenvolvimento da ciência contribuinte ao tema abordado e quanto aos desafios e avanços já conquistados.

No que tange a trabalhos relacionados, este trabalho possui características inéditas dentro do campo de QoS em SOC. Dentre as referências atuais e mais relevantes no que concerne modelos de QoS em SOC pode ser encontrado em \cite{pernici}, produzido pelo projeto europeu S-CUBE~\cite{scube}. Esta, além desse citado, produziu uma coletânea de outros relatórios e trabalhos que analisam publicações em praticamente todas as eferas do SOC. Entretanto, trata-se de trabalhos de Systematic Literature Review, visto que analisam profundamente as publicações envolvidas na área e as restringe àquelas com maior qualidade e aceitação, indicando as vantagens e limitações das propostas analisadas. Em contraste, o MS proposto abrange um número maior de estudos, trazendo informações em categorias mais amplas e que possibilitam a melhor análise geral da pesquisa relacionada. São dados amplos mas sensíveis para a compreensão do estado da ciência envolvida com os aspectos qualitativos do SOC.

Em geral, o resultado de um estudo de mapeamento é um mapa visual classificando os resultados obtidos. Em particular, acreditamos que esse mapeamento pode beneficiar o estado da arte e da pr\'{a}tica em Computa\c{c}\~{a}o Orientada a Servi\c{c}os identificando tend\^{e}ncias e oportunidades para transfer\^{e}ncia de conhecimento. Considerando tamb\'{e}m o tamanho e abrang\^{e}ncia dessa \'{a}rea outros  objetivos do presente mapeamento de estudos s\~{a}o tamb\'{e}m esclarecer o paradigma da orientação a serviços no contexto de qualidade de servi\c{c}os (QoS) por meio de uma classificação ampla e sistemática, obtendo informações sobre frequências de publica\c{c}\~{o}es, \'{a}reas e t\'{o}picos de pesquisa, enfoques, tipos de contribui\c{c}\~{o}es de pesquisa dadas, os agentes e f\'{o}runs envolvidos.

As demais se\c{c}\~{o}es desse artigo est\~{a}o organizadas da seguinte forma: Se\c{c}\~{a}o \ref{?}.Se\c{c}\~{a}o \ref{?}. Se\c{c}\~{a}o \ref{?}. Finalmente, na Se\c{c}\~{a}o \ref{?} apresentamos um resumo das nossas descobertas e provemos tamb\'{e}m algumas discuss\~{o}es finais. 

%Neste mapeamento, deve-se prezar pelo uso de ferramentas de apoio e que agilizem os procedimentos sistemáticos a serem seguidos. Não é seu objetivo avaliar qualitativamente os trabalhos de pesquisa classificados, mantendo a análise a um nível menos detalhado e que permitirá a inclusão dessas publicações em categorias abrangentes e significativas e que não exijam a minuciosa análise de cada uma delas. 

%Em relação aos atributos de QoS, tem-se por objetivo compreender que tipo de intervenções vem sendo feitas para a absorção de aspectos de qualidade em SOC e sua melhoria, identificando quais tipos de pesquisa,  de contribuição e quais atributos ou contextos tem maior importância, quais representam os maiores desafios para a concretização da adoção deste paradigma e quais são pouco abordados. Entre os variados e numerosos tipos de atributos que atendem a diversos modelos, este trabalho visa aqueles cuja definição encontra-se bem difundida e aceita entre os trabalhos relacionados. São eles o desempenho, a disponibilidade, a confiabilidade, a segurança, a modificabilidade, a testabilidade, a escalabilidade, o custo e outros para que os demais atributos sejam representados agrupadamente. Entre os tipos de pesquisa aceitos, estão os de avaliação, solução, validação, filosófico e de experiência pessoal. Por fim, entre os tipos de contribuições dadas, estão as de modelo, método, processo, ferramenta e métricas.
