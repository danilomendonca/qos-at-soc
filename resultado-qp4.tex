\section{Questão de Pesquisa 5}

\emph{Quais s\~{a}o os estudos existentes que mais tem impulsionado QoS em SOC?}

%Com essa quest\~{a}o de pesquisa, pretendemos tamb\'{e}m obter quais s\~{a}o os autores com maior n\'{u}mero de publica\c{c}\~{o}es relacionadas ao contexto desse estudo.

Identificamos quatro principais grupos de pesquisa que mais contribu\'{i}ram com pesquisas em SOC no contexto de QoS. Classificamos esses grupos conforme segue: 

\textbf{Grupo S-Cube} -- Identificamos que dos quatros grupos que mais se contribu\'{i}ram no \^{a}mbito desse estudo est\~{a}o pesquisadores cuja afilia\c{c}\~{a}o est\'{a} inserida direta ou indiretamente no contexto do grupo europeu S-Cube\~cite{SCube}. S\~{a}o os seguintes autores Schahram Dustdar (Vienna University of Techonology) e Raffaela Mirandola (Politecnico di Milano) com seus colaboradores Valeria Cardelini e Emiliano Casalicchio ambos de (Universit\'{a} di Roma ``Tor Vergata''). Em particular, percebemos que o Mirandola e seus colaboradores mais tem contribu\'{i}do em pesquisas relacionadas monitoramento e adapta\c{c}\~{a}o no contexto de confiabilidade, disponibilidade e desempenho como pode ser percebido com as publica\c{c}\~{o}es \cite{Cardellini:2009:QRA:1595696.1595718, Calinescu:2011:DQM:1990772.1991012, Ardagna:2010:POS:1814581.1814611, 10.1109/TSE.2011.68, Cardellini:2009:TSD:1692867.1692870}. Quanto a Dustdar e seus colaboradores, tem mais contribu\'{i}do em composi\c{c}\~{a}o de servi\c{c}os em ambientes din\^{a}micos no escopo de SLA, com destaque para o  VRESCO (\emph{Vienna Runtime Environment for Service-Oriented Computing})~\cite{5467022}. Em particular, uma observa\c{c}\~{a}o na IEEE Xplore mostra que essa contribui\c{c}\~{a}o recebeu at\'{e} o momento desse estudo, em torno de 42 cita\c{c}\~{o}es desde Setembro de 2010.

\textbf{Daniel Menasce et al.} -- Especial aten\c{c}\~{a}o para o SASSY~\cite{5696721} and Optimal Service Selection in SOA (also in collaboration with Casalicchio).


\textbf{Kwei-Jay Lin et al.} -- O grupo de Lin (University of Irvine) tem contribu\'{i}do essencialmente no contexto de SLA no \^{a}mbito de composi\c{c}\~{a}o (din\^{a}mica) e adapta\c{c}\~{a}o de servi\c{c}os. Particular destaque para  a contribui\c{c}\~{a}o \cite{Lin:2010:DIS:1861294.1861332} que lida com reconfigura\c{c}\~{a}o de servi\c{c}os devido a falhas, com restri\c{c}\~{o}es de QoS fim-a-fim no contexto de SOA~\cite{SOA REFERENCE}.

\textbf{Jie Xu et al.}  - University of Leeds
s
Emiliano Casalicchio have contributed both with Daniel M. and S-Cube group. 