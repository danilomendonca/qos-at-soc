\section{Questão de Pesquisa 4}

\emph{Qual o foco da contribuição de pesquisa realizada em SoC e relacionada com qualidade de serviço? }

Com o intuito de responder a esta questão, foi feita uma avalia\c c\~{a}o da distribui\c c\~{a}o dos 
artigos em rela\c c\~{a}o ao tipo de pesquisa (conforme discutido na Se\c c\~{a}o~\ref{sec:review_method}). 
Os \emph{bubble plots} nas Figuras~\ref{} e~\ref{}  apresentam tal distibui\c c\~{a}o, novamente sendo importante 
ressaltar que o n\'{u}mero total de artigos nos gr\'{a}ficos \'{e} superior ao n\'{u}mero total de artigos analisados--- 
uma vez que alguns artigos apresentam contribui\c c\~{o}es tanto em termos de uma nova solu\c c\~{a}o proposta 
quanto em termos de avalia\c c\~{a}o e/ou valida\c c\~{a}o. Por exemplo, Huang et al. prop\~{o}e um 
modelo estoc\'{a}stico para representar e raciocinar sobre dependabilidade em um ambiente de SoC, ao mesmo tempo 
que valida formalmente tal proposta por meio de provas de teoremas~\cite{huang:scc2011}.

Esta investiga\c c\~{a}o revelou que 21 artigos (como por exemplo~\cite{jeong:fqs2009,ardagna:jss2010}) 
contribuem com uma nova solu\c c\~{a}o para 
lidar com qualidade de servi\c co em SoC ao mesmo tempo que apresentam uma investiga\c c\~{a}o tanto 
em termos de avalia\c c\~{a}o quanto em termos de valida\c c\~{a}o. Al\'{e}m disso, 80 
artigos s\~{a}o propostas de solu\c c\~{o}es que apresentam 
avalia\c c\~{o}es em termos de estudos de casos mais simples, n\~{a}o compreendendo uma 
valida\c c\~{a}o da(s) t\'{e}cnica(s) proposta(s). Entre esses artigos podemos 
citar~\cite{filieri:faa2012, pernici:services2011,nascimento:splc2011}. Finalmente, 27 artigos apresentam, 
al\'{e}m de uma nova solu\c c\~{a}o relacionada \`{a} qualidade de servi\c cos em SoC, 
uma s\'{o}lida valida\c c\~{a}o (e.g.~\cite{huang:scc2011,binshtok:icsoc2009}).    
Por outro lado, 102 artigos (aproximadamente 40\% do total) prop\~{o}em novas 
solu\c c\~{o}es sem apresentar avalia\c c\~{a}o ou valida\c c\~{a}o consistentes (\cite{balfagih:icime2011,fiadeiro:fac2011,khazankin:scc2011}). 
Enquanto que apenas 4 artigos focam na avalia\c c\~{a}o e/ou valida\c c\~{a}o de 
propostas existentes~\cite{voelz:edoc2010,moayed:icsea2010,cavallo:pesos2010,banerjee:isorcw2011}.  

Esses n\'{u}meros revelam que a \'{a}rea de pesquisa de qualidade de servi\c co em 
SoC ainda est\'{a} em uma fase 
de amadurecimento, onde um percentual significativo das contribui\c c\~{o}es 
simplesmente apresentam novas abordagens ou fazem compara\c c\~{o}es envolvendo a 
pr\'{o}pria t\'{e}cnica proposta--- o que pode levar a conclus\~{o}es tendenciosas. A quantidade 
de artigos que visam avaliar ou validar t\'{e}cnicas existentes, algo recomendado antes 
de se iniciar a concep\c c\~{a}o de uma nova solu\c c\~{a}o, \'{e} praticamente insignificante. 
 
